\documentclass[12pt,fleqn]{article}
\usepackage[margin=1in,top=1in,bottom=1in]{geometry}
\usepackage{mathtools}
\usepackage{longtable}
\usepackage{enumitem}
%\usepackage{hyperref}
\usepackage[dvips]{graphics}
\usepackage[table]{xcolor}
\usepackage{amssymb}
%\usepackage{subfig}
\usepackage{booktabs}
\usepackage{tikz}
\usepackage{subcaption}

\usepackage[normalem]{ulem}

\usepackage{multicol}
\usepackage{txfonts}
%\usepackage{amsfonts}

%%%%%%%%%% bibliography stuff %%%%%%%%%%%%%
\usepackage[square,numbers]{natbib}
\bibliographystyle{abbrvnat}
%\usepackage{natbib}
%\bibliographystyle{/Users/tonhauser.1/Library/Latex/cslipubs-natbib}

\setlength{\bibhang}{0.5in}
\setlength{\bibsep}{0mm}
%\bibpunct[:]{(}{)}{;}{a}{,}{,}
%%%%%%%%%%%%%%%%%%%%%%%%%%%%%%%

\usepackage{wrapfig}

\usepackage{gb4e}
%\usepackage{/Users/judith/Library/Latex/drs}
%\usepackage{/Users/judith/Library/Latex/avm}
\usepackage[all]{xy}
\usepackage{rotating}
\usepackage{tipa}
\usepackage{multirow}
\usepackage{authblk}
\usepackage{adjustbox}
\usepackage{array}

\usepackage{titlesec}
\titleformat*{\section}{\bfseries\footnotesize}
 
\setlength{\parindent}{.3cm}
\setlength{\parskip}{0ex}

\renewcommand\figurename{Fig.}

\newcommand{\yi}{\'{\symbol{16}}}
\newcommand{\nasi}{\~{\symbol{16}}}
\newcommand{\hina}{h\nasi na}
\newcommand{\ina}{\nasi na}

\exewidth{(\thexnumi)}

\newcommand{\citepos}[1]{\citeauthor{#1}'s \citeyear{#1}}

\newcommand{\6}{\mbox{$[\hspace*{-.6mm}[$}} 
\newcommand{\9}{\mbox{$]\hspace*{-.6mm}]$}}
\newcommand{\sem}[2]{\6#1\9$^{#2}$}
\renewcommand{\ni}{\~{\i}}

\newcommand{\jt}[1]{\textbf{\color{blue}JT: #1}}


\setlength{\belowcaptionskip}{-10pt}


 \begin{document}
 
%The main text should be at most 2 pages (US Letter or A4) in length, including examples, with an optional 3rd page for references or also large graphs, tables, pictures and figures.
 
\begin{center}
{\bf An empirical challenge to the categorical notion of factivity}

Judith Tonhauser (OSU / Stuttgart U) \& Judith Degen (Stanford U) 
\end{center}

\vspace*{-.2cm}

%\begin{wrapfigure}{r}{0.32\textwidth}
%\centering
%  \includegraphics[trim={3.5cm 1cm 7cm 2cm},clip,width=.27\paperwidth]{../paper/figures/categorization}
%  \caption{Standard classification of 20 English predicates}\label{f-cat}
%\end{wrapfigure} 

\noindent
Projection analyses have largely limited their attention to `factive' predicates, like {\em know}, to the exclusion of `non-factive' ones, like {\em think}  (e.g., \cite{heim83,vds92,abrusan2011,abrusan2016,romoli2015,best-question}). This limitation is motivated by the long-standing and widely-held assumption that `factive' predicates are empirically distinct from `non-factive' ones (see, e.g., \cite{karttunen71b,kiparsky-kiparsky71} and much literature thereafter): the content of the complement (CC) of a `factive' predicate is taken to be both projective and entailed, whereas `non-factive' CCs  are taken to not be both projective and entailed  (e.g., \cite{gazdar79a}, \cite{ccmg90}, \cite{vds92},  \cite{schlenker10}, \cite{anand-hacquard2014}, \cite{spector-egre2015}). 

Despite the importance of the distinction between `factive' and `non-factive' predicates, the distinction has not been systematically investigated. Filling this lacuna is particularly pressing given that there is disagreement about which predicates are `factive'. For instance, \cite{schlenker10} assumed that {\em inform} is `factive', in contrast to \cite{anand-hacquard2014} who argued that \emph{inform}'s CC is not entailed. Similarly, emotive predicates like {\em be annoyed} are taken to be `factive' by some (e.g., \cite{gazdar79a,abrusan2011,anand-hacquard2014}) but not others (e.g., \cite{klein1975,giannakidou1998,schlenker2003,egre2008}). We report the results of two experiments designed to investigate the distinction by measuring projectivity and entailment for 20 predicates (see left panel of Fig.\ \ref{f-summary-categorical}). %{\bf maybe add question marks in the figure after the predicates where people have disagreements? be annoyed, inform, establish, hear)}

{\bf Exp.~1 (n=300):} Projectivity was measured with the `certain that' diagnostic. For 20 items like ``Helen asks: \emph{Did Amanda discover that Danny ate the last cupcake?}'' participants rated the speaker's certainty about $p$ (that Danny ate the last cupcake) on a sliding scale from \emph{no} to \emph{yes}. Higher certainty ratings indicate greater projectivity. Following \cite*{tbd-variability}, we assume that projectivity is a gradient property of utterance content. 

{\bf Exp.~2 (n=300):} Entailment was measured via the `inference' diagnostic: $p$ entails $q$ iff $q$ follows from the truth of $p$. %; under the `contradictoriness' diagnostic, $p$ entails $q$ iff $p$ {\em but not} $q$ is contradictory. 
Participants rated 20 items like ``What is true: Amanda discovered that Danny ate the last cupcake.'' for whether it follows that Danny ate the last cupcake. Predicates were classified as entailing iff they behaved no differently from entailing control stimuli. 

{\bf Results and discussion:} As shown in the right panel of Fig.\ \ref{f-summary-categorical}, the CCs of `factive' predicates (in purple) tend to be entailed and highly projective whereas those of classical `non-factive' predicates (in grey) are not entailed and at most weakly projective, in line with intuitions reported in the literature. However, because projectivity is again found to be a gradient property of utterance content (see also \cite{tbd-variability}), there is no non-arbitrary binary division of predicates by projectivity,  thereby challenging the assumed categorical distinction between `factive' and `non-factive' predicates. 
\vspace{-1em}
\begin{wrapfigure}{r}{.77\textwidth}
\begin{subfigure}{.35\textwidth}
\centering
\includegraphics[width=.27\paperwidth]{../paper/figures/categorization}
%\caption{Assumed categorical classification}
\end{subfigure} %
\begin{subfigure}{.3\textwidth}
\centering
\includegraphics[width=.3\paperwidth]{../results/5-projectivity-no-fact/graphs/projection-by-inferenceEntailment}
%\caption{Experiment findings}
\end{subfigure}

\caption{Received categorical classification (left) vs.\ experimental results (right; entailment against mean certainty ratings) for 20 predicates.}\label{f-summary-categorical}

\end{wrapfigure}

\noindent
We discuss the exciting challenge for future projection analyses posed by the result that the CC of many `non-factive' predicates is projective. 





%\begin{figure}[h]
%
%\begin{subfigure}{.35\textwidth}
%\centering
%\includegraphics[trim={3.5cm 1cm 7cm 2cm},clip,width=.27\paperwidth]{../paper/figures/categorization}
%%\caption{Assumed categorical classification}
%\end{subfigure} %
%\begin{subfigure}{.3\textwidth}
%\centering
%\includegraphics[width=.3\paperwidth]{../results/5-projectivity-no-fact/graphs/projection-by-inferenceEntailment}
%%\caption{Experiment findings}
%\end{subfigure}
%
%\caption{20 English predicates: assumed categorical classification (left panel) vs.\ experiment findings (right panel; mean certainty rating with 95\% CIs, inference diagnostic for entailment.}\label{f-summary-categorical}
%
%\end{figure}

\newpage

\textbf{JT, can you add "(higher=more projective)" to the x axis label in the right panel?}

{\bf (20\% mis-classified on Exp 2a, 6/20 on Exp 2b)}

{\bf somehow mention second entailment diagnostic} research on entailment needs to take the pragmatics of entailment judgments more seriously (see also \cite{demarneffe-etal2012}). 

\bibliography{../bibliography}

\end{document}
