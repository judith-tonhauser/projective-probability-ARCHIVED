\documentclass[11pt,fleqn]{article}
\usepackage[margin=1in,top=1in,bottom=1in]{geometry}
\usepackage{tikz}
\usepackage{mathtools}
\usepackage{longtable}
\usepackage{enumitem}
\usepackage{hyperref}
%\usepackage[dvips]{graphics}
%\usepackage[table]{xcolor}
%\usepackage{amssymb}
\usepackage{float}
%\usepackage{subfig}
\usepackage{booktabs}
\usepackage{subcaption}

\usepackage[normalem]{ulem}

\usepackage{multicol}
\usepackage{txfonts}
\usepackage{amsfonts}
\usepackage{natbib}
\usepackage{gb4e}
\usepackage[all]{xy}
\usepackage{rotating}
\usepackage{tipa}
\usepackage{multirow}
\usepackage{authblk}
\usepackage{url}
\usepackage{pdflscape}
\usepackage{rotating}
\usepackage{adjustbox}
\usepackage{array}

\def\bad{{\leavevmode\llap{*}}}
\def\marginal{{\leavevmode\llap{?}}}
\def\verymarginal{{\leavevmode\llap{??}}}
\def\swmarginal{{\leavevmode\llap{4}}}
\def\infelic{{\leavevmode\llap{\#}}}

\definecolor{airforceblue}{rgb}{0.36, 0.54, 0.66}

\newcommand{\dashrule}[1][black]{%
  \color{#1}\rule[\dimexpr.5ex-.2pt]{4pt}{.4pt}\xleaders\hbox{\rule{4pt}{0pt}\rule[\dimexpr.5ex-.2pt]{4pt}{.4pt}}\hfill\kern0pt%
}

\setlength{\parindent}{.3in}
\setlength{\parskip}{0ex}

\newcommand{\yi}{\'{\symbol{16}}}
\newcommand{\nasi}{\~{\symbol{16}}}
\newcommand{\hina}{h\nasi na}
\newcommand{\ina}{\nasi na}

\newcommand{\foc}{$_{\mbox{\small F}}$}

\hyphenation{par-ti-ci-pa-tion}

\setlength{\bibhang}{0.5in}
\setlength{\bibsep}{0mm}
\bibpunct[:]{(}{)}{,}{a}{}{,}

\newcommand{\6}{\mbox{$[\hspace*{-.6mm}[$}} 
\newcommand{\9}{\mbox{$]\hspace*{-.6mm}]$}}
\newcommand{\sem}[2]{\6#1\9$^{#2}$}
\renewcommand{\ni}{\~{\i}}

\newcommand{\citepos}[1]{\citeauthor{#1}'s \citeyear{#1}}
\newcommand{\citeposs}[1]{\citeauthor{#1}'s}
\newcommand{\citetpos}[1]{\citeauthor{#1}'s (\citeyear{#1})}

\newcolumntype{R}[2]{%
    >{\adjustbox{angle=#1,lap=\width-(#2)}\bgroup}%
    l%
    <{\egroup}%
}
\newcommand*\rot{\multicolumn{1}{R{90}{0em}}}% no optional argument here, please!


\title{Factive versus non-factive predicates? An empirical investigation of the lexical meaning of clause-embedding predicates
\thanks{Acknowledgments removed for review process}}

%\thanks{For helpful comments on the research presented here, we thank David Beaver, Kai von Fintel, Mandy Simons, the anonymous reviewers for {\em Semantics and Linguistic Theory} 2018, as well as the audiences at the MIT Linguistics colloquium and the 2018 Annual Meeting of XPRAG.de. We gratefully acknowledge financial support for this research from {\em National Science Foundation} grant BCS-1452674 (JT) and the Targeted Investment for Excellence Initiative at The Ohio State University (JT).}}

\author{Author(s)}

%\author[$\circ$]{Judith Tonhauser}
%\author[$\bullet$]{Judith Degen}

%\affil[$\circ$]{The Ohio State University / University of Stuttgart}
%\affil[$\bullet$]{Stanford University}

%\renewcommand\Authands{ and }

\newcommand{\jt}[1]{\textbf{\color{blue}JT: #1}}

\begin{document}

%\tableofcontents
%\newpage

\maketitle


\begin{abstract}

Factive and non-factive predicates have long been assumed to be categorically distinct: the content of the complement of a factive predicate is presupposed, whereas that of non-factive predicates is not (e.g., \citealt{karttunen71b,kiparsky-kiparsky71,heim83,vds92,abusch10,abrusan2011,abrusan2016,romoli2015,best-question}. Analyses of the content of the complement of factive predicates as presupposed account for the observation that this content may project over entailment-canceling operators. However, the assumption that factive and non-factive predicates are categorically distinct is challenged by the observation that the content of the complement of some non-factive predicates may project (e.g., \citealt{schlenker10,anand-hacquard2014,spector-egre2015}) and that there is projection variability among factive predicates (\citealt{karttunen71b,tbd-variability}). Furthermore, there is disagreement in the literature about whether the content of the complement of predicates assumed to be factive is entailed, i.e., whether the predicates are indeed factive (e.g., \citealt{klein1975,hazlett2010,abrusan2011}). 


A `factive' predicate is standardly defined as one for which the content of the clausal complement is both projective and entailed, like {\em know} or {\em discover}, whereas it is not both projective and entailed for `non-factive' predicates, like {\em think} or {\em announce}. Based on the findings of experiments designed to explore projectivity and entailment with 20 `factive' and `non-factive' English clause-embedding predicates, this paper shows that `factive' predicates are more heterogeneous than previously assumed and that, contrary to assumption, the categorical distinction between `factive' and `non-factive' predicates is not empirically supported. We conclude that the empirical purview of projection analyses is broader than previously assumed and that, in particular, it is not justified for formal analyses of the projectivity of the content of the complement of clause-embedding predicates to be limited to `factive' predicates (see, e.g., \citealt{heim83,vds92,abrusan2011,abrusan2016,romoli2015} and \citealt{best-question}).


\end{abstract}

-- Early literature (Kiparsky\&Kiparsky) -- clause embedding verbs fall into two classes, factive \& non-factive, semantically distinguished in whether or not they presuppose truth of complement (presupposition treated as semantic property) Presupposition assumed to be diagnosable by projection, so assume that complements of factives project, complements of non-factives don't project. 
-- More recent literature: increasing recognition that projection is not restricted to presuppositional contents (see Potts on conventional implicature, refs discussing projection associated with non-presuppositional vbs i.e. non-factive projective). At the same time, we have increasing evidence that projection is not all-or-nothing, but comes in degrees (cite Judith, Judith, David)
-- Raises the question: what features affect the projection of clausal complements?

-- This is what the CommitmentBank is intended to investigate.

Thoughts?


On another note: The predicates identified as factive in K\&K 1970 clearly do not all entail their complements (e.g. take into account, bear in mind, make clear). So I would really like to know at what point in the literature the idea that entailment is part of the characterization of factivity first appeared.
			
\section{Introduction}\label{s1}

Based on experimental evidence, this paper challenges the long-standing and widely-adopted distinction between `factive' and `non-factive' clause-embedding predicates (see, e.g.,\citealt{karttunen71b,kiparsky-kiparsky71} and much literature thereafter). It is standardly assumed, as defined in (\ref{def}), that the content of the complement of `factive' predicates, such as {\em know} or {\em discover}, is both entailed and projective, whereas the content of the complement of `non-factive' predicates, like {\em think} or {\em inform},  is not both entailed and projective (e.g., \citealt[119-123]{gazdar79a}, \citealt[355]{ccmg90}, \citealt[345]{vds92},  \citealt[3]{abbott06}, \citealt[139]{schlenker10}, \citealt[77]{anand-hacquard2014}, \citealt[fn.7]{spector-egre2015}).

\begin{exe}
\ex\label{def} Let $S$ be an unembedded matrix sentence in which a finite clause $C$ is the argument of a clause-embedding predicate $P$. The predicate $P$ is {\bf factive} if and only if the content of $C$ is

\begin{xlist}
\ex entailed by the content of $S$ and
\ex projective in utterances of family-of-sentence variants of $S$.
\end{xlist}
Otherwise, the predicate is {\bf non-factive}.
\end{exe}

To illustrate the standard assumption, consider the examples with the `factive' predicate {\em know} in  (\ref{know}) and (\ref{know2}), respectively. (\ref{know}) is typically taken to entail the content of the complement of {\em know}, that Julian dances salsa. The so-called family-of-sentences variants of (\ref{know}) given in (\ref{know2}a-d) do not entail this content because {\em know} and its clausal complement occur in an entailment-canceling environment: under negation in (\ref{know2}a), in a polar question in (\ref{know2}b), under the epistemic possibility modal {\em perhaps} in (\ref{know2}c) and in the antecedent of a conditional in (\ref{know2}d). Since speakers who utter the sentences in (\ref{know2}a-d) may nevertheless be taken to be committed to the truth of the content of the complement, this content, by virtue of being able to `project' over the entailment-canceling operators, is projective content (e.g., \citealt{potts05,brst-salt10}). 

\begin{exe}

\ex\label{know} Sandra knows that Julian dances salsa.

\ex\label{know2} 

\begin{xlist} 
\ex Sandra doesn't know that Julian dances salsa. 
\ex Does Sandra know that Julian dances salsa?
\ex Perhaps Sandra knows that Julian dances salsa.
\ex If Sandra knows that Julian dances salsa, she'll ask him to dance. 
\end{xlist}

\end{exe}

In contrast, the variant of (\ref{know}) with the `non-factive' predicate {\em think} in (\ref{think}) does not entail the content of the complement.  Similarly, the content of the complement of {\em think} is not generally taken to be projective: speakers who utter one of the family-of-sentences variants of (\ref{think}) in (\ref{think2}) are not typically taken to be committed to the truth of the content of the complement. 

\begin{exe}

\ex\label{think} Sandra thinks that Julian dances salsa.
\ex\label{think2} 

\begin{xlist} 
\ex Sandra doesn't think that Julian dances salsa. 
\ex Does Sandra think that Julian dances salsa?
\ex Perhaps Sandra thinks that Julian dances salsa.
\ex If Sandra thinks that Julian dances salsa, she'll ask him to dance. 
\end{xlist}

\end{exe}

Given the definition in (\ref{def}), `factive' predicates are a homogenous class of predicates: the content of the complement is both entailed and projective. `Non-factive' predicates, on the other hand, are heterogeneous. For a first subset, henceforth called `plain non-factives', including {\em think} and {\em pretend}, the content of the complement is neither entailed nor projective, as illustrated in (\ref{think}) and (\ref{think2}) for {\em think}. For a second subset, including {\em be right} and {\em demonstrate}, the content of the complement is entailed but not projective; we refer to these predicates as `veridical non-factives'. For instance, the content of (\ref{right}) is generally taken to entail the content of the complement, but speakers who utter one of the family-of-sentences variants in (\ref{right2}) are not taken to be committed to the content of the complement, which means that it is not projective. 

\begin{exe}
\ex\label{right} Sandra is right that that Julian dances salsa.
\ex\label{right2} 

\begin{xlist} 
\ex Sandra isn't right that Julian dances salsa. 
\ex Is Sandra right that Julian dances salsa?
\ex Perhaps Sandra is right that Julian dances salsa.
\ex If Sandra is right that Julian dances salsa, she'll ask him to dance. 
\end{xlist}
\end{exe}
Finally, for a third subset of `non-factive' predicates, including {\em announce, tell, confess, establish, inform, acknowledge, admit} and {\em confirm}, the content of the complement is projective but not entailed; for comments to this effect see, e.g., \citealt{reis1973,melvold1991,schultz2003,swanson2012,anand-hacquard2014,spector-egre2015,karttunen2016,tbd-variability}.  We refer to these `non-factive' predicates as `projective non-factive'. To illustrate, consider the sentence in (\ref{announce2}a) with {\em announce}. As pointed out by \citet[139]{schlenker10}, if Mary is a responsible 30-year old woman, then speakers who utter (\ref{announce2}a) may very well be taken to be committed to the content of the complement, that Mary is pregnant, i.e., this content is projective. That the content of the complement of {\em announce} is not entailed, and {\em announce} therefore is a `non-factive' predicate, can be seen from the fact that Mary being pregnant does not necessarily follow from (\ref{announce2}b), for instance if Mary is a 7-year old girl.

\begin{exe}
\ex\label{announce2} 
\begin{xlist}
\ex Has Mary announced to her parents that she is pregnant? 
\ex Mary has announced to her parents that she is pregnant. \hfill (\citealt[139]{schlenker10})
\end{xlist}
\end{exe}
Similarly, \citet{spector-egre2015} suggested that it is possible to take a speaker who utters (\ref{tell}a) with the `non-factive' predicate {\em tell}  to be committed to the content of the complement, that Fred is the culprit. That this content is not entailed, and {\em tell} therefore is a `non-factive' predicate, is shown by the fact that the content does not follow from the unembedded variant in (\ref{tell}a) if Sue is known to be an unreliable gossip.

\begin{exe}
\ex\label{tell}
\begin{xlist}
\ex Sue didn't tell Jack that Fred is the culprit. 
\ex Sue told Jack that Fred is the culprit. \hfill (\citealt[1739]{spector-egre2015})
\end{xlist}
\end{exe}

In sum, entailment and projectivity are assumed to jointly distinguish `factive' predicates from `non-factive' ones: as shown in Figure \ref{f-assumption}, the content of their complement of `factive' predicates is defined to be both entailed and projective, whereas that of `non-factive' predicates is not both entailed and projective. 

\begin{figure}[h]
\center

\begin{tikzpicture}

% x and y axes
\draw[-,thick] (0,0) -- (5,0);
\draw[-,thick] (0,0) -- (0,5);

% labels on the axes
\node at (2.5,-.7) {{\bf projective}};
\node at (1,-.3) {no};
\node at (4,-.3) {yes};

\node at (-1.7,2.5) {{\bf entailed}};
\node at (-.5,4) {yes};
\node at (-.5,1) {no};

% predicates
\node at (1,1) {{\em think}};
\node at (1,1.4) {{\em pretend}};
\node at (1,.6) {{\em doubt}};

\node at (1,3.6) {{\em be right}};
\node at (1.2,4) {{\em demonstrate}};
\node at (1,4.4) {{\em be true}};

\node at (4,.6) {{\em announce}};
\node at (4,1) {{\em tell}};
\node at (4,1.4) {{\em confess}};

\node at (4,3.6) {{\em know}};
\node at (4,4) {{\em discover}};
\node at (4,4.4) {{\em regret}};

% factive rectangle
\node[blue, thick] at (4,5.3) {factive};
\draw[blue, very thick] (2.7,2.7)rectangle (5,5);


% non-factive rectangle
\node[green, thick] at (1.2,5.3) {non-factive};
\draw[green, very thick] (0.1,0.3) -- (5,0.3); %bottom horizontal line
\draw[green, very thick] (0.1,5) -- (2.3,5); %top horizontal line
\draw[green, very thick] (0.1,0.3) -- (0.1,5); %leftmost vertical line
\draw[green, very thick] (2.3,5) -- (2.3,2); %middle vertical line
\draw[green, very thick] (2.3,2) -- (5,2); %middle horizontal line
\draw[green, very thick] (5,2) -- (5,0.3); %rightmost vertical line

\end{tikzpicture}

\caption{Illustration of the standardly-assumed division of English clause-embedding predicates into `factive' and `non-factive' ones based on whether the content of the clausal complement is entailed or projective}\label{f-assumption}
\end{figure}

The distinction between `factive' and `non-factive' predicates has played a critical role in formal analyses of the projectivity of the content of the complement of clause-embedding predicates.\footnote{The distinction has also been invoked in analyses of the interpretation of embedded questions, as in {\em John knows whether it is raining}. For instance, \citet{spector-egre2015} proposed that `factive' predicates and those for which the content of the complement is merely entailed are those that express a relation to the actual true answer, as opposed to a potential answer. Our findings are thus also relevant for such analyses, though we limit the discussion to implications for projection analyses.} First, the empirical purview of such analyses has been defined by the distinction: to date, formal analyses are largely limited to the content of the complement of `factive' predicates (some exceptions will be discussed later), despite the fact that the content of the complement of some `non-factive' predicates is also projective. This is primarily due to the fact that early analyses, like that of \citealt{heim83} and \citealt{vds92}, were limited to the content of the complement of `factive' predicates. Specifically, in these `lexicalist' analyses, the content of the complement of `factive' predicates is lexically specified as a presupposition, thereby predicting that the content of the complement is projects over entailment-canceling operators. But because the lexical specification of the content of the complement as a presupposition also predicts that the content follows from unembedded matrix sentences, lexicalist analyses were limited to `factive' predicates, for which the content of the complement is both projective and entailed. 

A second way in which the distinction between `factive' and `non-factive' predicates has played a critical role in formal projection analyses is that more recent analyses derive the projectivity of the content of the complement of `factive' predicates from its status as entailed content (e.g., \citealt{abrusan2011,abrusan2016}, \citealt{romoli2015,best-question}). For instance, on Abrus\'an's (2011, 2016) analysis, entailed content is by default backgrounded and therefore projective, and on \citetpos{best-question} analysis, questions addressed by utterances of sentences that entail the content of the complement may also entail that content, which is thereby projective. Thus, in contrast to lexicalist analyses, on which entailment and projectivity jointly define content that is lexically specified to be a presuppositions, entailment is a necessary and sufficient condition to deriving projectivity on these `entailment-based' analyses. (As discussed below, these analyses do not typically consider predictions made for `veridical non-factive' predicates.) In short, the distinction between `factive' predicates, for which the content of the complement is both entailed and projective, and `non-factive' ones, for which it is not, has critically shaped the empirical purview and nature of  formal projection analyses. 

In this paper, we investigate the assumed categorical distinction between `factive' and `non-factive' predicates on the basis of experiments designed to explore entailment and projectivity. The impetus for our investigation came from observations in prior literature that challenge the assumed distinction. Consider first projectivity. It has long been recognized that there is variability in the projectivity of the content of the complement of `factive' predicates. \citet{karttunen71b}, for instance, already suggested that the content of the complement of {\em regret} in (\ref{semi-factive}a) is more projective than that of {\em discover} in (\ref{semi-factive}b): 


\begin{exe}
\ex\label{semi-factive}
\begin{xlist}
\ex John didn't regret that he had not told the truth.
\ex John didn't discover that he had not told the truth.  
\hfill (\citealt[63]{karttunen71b})

\end{xlist}
\end{exe}
More recently, \citet*{tbd-variability} showed that there is even more projection variability than previously assumed, including among presumed `factive' predicates. In particular, their Experiment 1b showed that the content of the complement of {\em be annoyed, be amused, notice, be aware, realize, see, find out} and {\em learn} was more projective than that of {\em reveal}, which in turn was more projective than that of {\em discover}.  Because the definition of `factive' and `non-factive' predicates in (\ref{def}) assumes a categorical division between projective and non-projective content, the observed projection variability raises the question of whether a distinction between projective and non-projective content can be maintained and, if yes, how projective the content of the complement of a clause-embedding predicate needs to be in order for the predicate to be considered `factive'. In this paper, we systematically explore the projectivity of the content of the complement of a range of English predicates presumed to be `factive' and `non-factive'. 

Disagreements in the prior literature about whether the content of the complement is entailed also provide reason to systematically explore the assumed division between `factive'  and `non-factive' predicates. \citet[139]{schlenker10}, for example, assumed that the content of the complement of {\em inform} is entailed, i.e., that Mary being pregnant necessarily follows from (\ref{inform}).  

\begin{exe}
\ex\label{inform} Mary has informed her parents that she's pregnant.

%\ex Mary has announced to her parents that she's pregnant.
\end{exe}
\citet[76]{anand-hacquard2014}, however, pointed out that {\em inform} can be modified by {\em falsely}, as in the naturally occurring example in (\ref{inform2}): the content of the complement of {\em inform}, that the victims had won more than a million dollars in a lottery, does not follow from (\ref{inform2}). As a consequence, they argue, the content of the complement of {\em inform}  is not entailed and {\em inform} is not a `factive' but a `non-factive' predicate.

\begin{exe}
\ex\label{inform2} From March 2012, Peart's and King's co-conspirators are alleged to have [...] falsely informed [victims] that they had won more than a million dollars in a lottery. \hfill (\citealt[76]{anand-hacquard2014})
\end{exe}
Similarly, both \citet{wyse} and \citet{swanson2012} took the content of the complement of {\em establish} to be entailed, but the naturally occurring example in (\ref{establish}), in which {\em establish} is  modified by {\em falsely}, again shows that this content does not necessarily follow from unembedded matrix sentences with {\em establish}. In other words, (\ref{establish}) suggests that {\em establish} is a `non-factive' predicate.

\begin{exe}
\ex\label{establish} The testimony is uncontradicted that Jew Sing committed at least six perjuries and a subornation of perjury to conceal his wrongful entry into the country, and falsely established that he was an American citizen. \hfill \begin{footnotesize}\url{https://law.justia.com/cases/federal/appellate-courts/F2/202/715/216831}\end{footnotesize}
\end{exe}

There is also disagreement about whether the content of the complement of emotive `factive' predicates, like {\em be annoyed} or {\em regret}, is entailed. Consider the examples in (\ref{emotive1}), in which two standard diagnostics for entailment are applied to the content of the complement of {\em regret}. First, as noted in \citealt[514]{abrusan2011}, ``from [(\ref{emotive1}a)] one tends to infer that it is raining'', suggesting that the content of the complement is entailed. Second, a speaker who utters the sentence in (\ref{emotive1}b) is judged to contradict themselves (as indicated by the hashmark): it does not seem possible for the speaker to be committed to the proposition that it is not raining and to the proposition that John regrets that it is raining. Examples like these support the position that the content of the  complement of emotive predicates is entailed, as assumed in, e.g., \citealt{gazdar79a,abrusan2011} and \citealt{anand-hacquard2014}.

\begin{exe}

\ex\label{emotive1} \citealt[514]{abrusan2011}

\begin{xlist}

\ex I doubt that John regrets that it's raining.

\ex \infelic It's not raining but John regrets that it's raining. 

\end{xlist}

\end{exe}

There are, however, also examples that show that the content of the complement of emotive predicates does not necessarily follow from unembedded matrix sentences. In (\ref{emotive2}a), the content of the complement of {\em regret}, that Oedipus killed the stranger on the road to Thebes, does not follow and, in (\ref{emotive2}b), the content of the complement of {\em be enraged}, that Mary is no longer single, does not follow. The existence of such examples has led some authors, including \citet{klein1975,giannakidou1998,schlenker2003} and \citet{egre2008}, to conclude that the content of the complement of emotive predicates is not entailed, i.e., that they are not `factive' predicates by the definition in (\ref{def}).

\begin{exe}

\ex\label{emotive2}

\begin{xlist}

\ex Falsely believing that he had inflicted a fatal wound, Oedipus regretted killing the stranger on the road to Thebes \hfill (\citealt{klein1975})

\ex John wrongly believes that Mary got married, and he is enraged that she is no longer single. \\ \hspace*{.2cm} \hfill (adapted from \citealt{egre2008}, based on \citealt{schlenker03})

\end{xlist}

\end{exe}
Emotive predicates are not the only `factive' predicates that can be used in discourses in which the content of the complement does not necessarily follow. Consider the examples in (\ref{fis}) with the cognitive `factive' predicate {\em know} from  \citealt{hazlett2010} and \citealt{abrusan2011}:  it does not follow from (\ref{fis}a) that ulcers are caused by bacterial infection or from (\ref{fis}b) that the police are listening to John's phone calls.\footnote{Some authors, including \citet{gazdar79a} and \citet{abrusan2011}, do not take examples like (\ref{emotive2}) or (\ref{fis}) to show that the content of the complement is not entailed. They instead suggested that such examples involve free indirect discourse or are understood with a different concept of knowledge. We discuss these proposals in section \ref{s3} in the context of the findings of our entailment experiments.}

\begin{exe}
\ex\label{fis}
\begin{xlist}

\ex Everyone knew that stress caused ulcers, before two Australian doctors in the early 80s proved that ulcers are actually caused by bacterial infection. \hfill (\citealt[501]{hazlett2010})

\ex John suffers from paranoia. He falsely believes that the police is spying on him and what is more he knows they are listening to his phone calls. \hfill (\citealt[514]{abrusan2011})

%\ex The keys were not in the drawer but she knew that they were there, so she foolishly kept on searching. \hfill (\citealt[514]{abrusan2011})

%\ex In school we learned that WWI was a war ``to make the world safe for democracy," when it was really a war to make the world safe for the Western imperial powers. \hfill (\citealt[501]{hazlett2010})

\end{xlist}

\end{exe}

In sum, there is disagreement in the literature about which clause-embedding predicates are such that the content of their complement is entailed.\footnote{\citealt{sudo-thesis} proposed that not all presuppositions are entailments based on the observation that not all presuppositions obligatorily restrict the domain of the quantifier {\em exactly one}. For instance, the pre-state content of {\em stop} is a presupposition and an entailment because (ia) means that exactly one student who used Mac stopped using Mac, but the gender presupposition of {\em she} in (ib) is not an entailment because (ib) does not mean that exactly one female student criticized herself, but that exactly one student criticized herself (see \citealt{zehr-schwarz2016,zehr-schwarz2018} for some experimental support). According to this diagnostic, the content of the complement of {\em know} is entailed content: (ic) means that exactly one student who got accepted by MIT knows that they got accepted by MIT. Whether the {\em exactly one} diagnostic can help settle the debate about which contents of complements of clause-embedding predicates are entailed is a question we leave for future research.

\begin{exe}
\exi{(i)} 
\begin{xlist}
\ex Exactly one student stopped using Mac. \hfill (\citealt[59]{sudo-thesis})

\ex Exactly one student criticized herself. \hfill (\citealt[61]{sudo-thesis})

\ex Exactly one student knows that they got accepted by MIT. \hfill (\citealt[64]{sudo-thesis})

\end{xlist}
\end{exe}} Despite the central importance of entailment in the identification of `factive' predicates, there has, to date, not been any systematic investigation of the question of which predicates are such that the content of the clausal complement is entailed.\footnote{\citet{djaerv-etal2016} set out to investigate whether emotive and cognitive `factive' predicates differ with respect to whether the content of the clausal complement is entailed. However, because the `factive' predicates investigated in their experiments were realized in an entailment-canceling environment, namely polar questions (e.g., {\em Is Maria aware that Mike is moving back to Chicago?}), the content of the complement is not entailed in their experiment stimuli. Consequently, the observed differences cannot directly bear on the question of whether the two types of predicates differ in whether the content of the complement is entailed.} This paper fills this empirical lacuna by exploring whether the content of the complement of a broad range of English predicates presumed to be `factive' and `non-factive' is entailed.

Our experimental exploration of projectivity and entailment for English clause-embedding predicates serves to assess whether the long-standing and widely-assumed division between `factive' and `non-factive' predicates is empirically supported. In section \ref{s2}, we present the findings of Experiment 1, which was designed to explore the projectivity of the content of the complement of 20 English clause-embedding `factive' and `non-factive' predicates.%\footnote{\label{f-github}The experiments, data and R code for generating the figures and analyses of the experiments reported on in this paper are available at \url{https://github.com/judith-tonhauser/factivity}.}  
In section \ref{s3}, two standard diagnostics for entailment are applied in Experiments 2a and 2b, respectively, to the content of the complement of the same 20 clause-embedding predicates. On the basis of the findings from Exps.~1 and 2, we show in section \ref{s4} that `factive' predicates are more heterogeneous than previously assumed and argue that, contrary to assumption, the categorical distinction between `factive' and `non-factive' predicates is not empirically supported. We conclude that  the empirical purview of projection analyses is broader than previously assumed and that, in particular, it is not empirically justified for analyses of the projectivity of the content of the complement of clause-embedding predicates to be limited to `factive' predicates.

 %\begin{exe}
%\ex\label{swan} She confessed that she took the money, but later recanted. It turned out that she had been trying to cover up a friend's mistake. \hfill (adapted from \citealt[1540]{swanson2012})
%\end{exe}


\section{Experiment 1: Projectivity}\label{s2}

Exp.~1 was designed to explore the projectivity of the content of the complement of 20 English clause-embedding predicates. Following \citealt{tbd-variability}, we assume that projectivity is a gradient property of utterance content: an utterance content projects to the extent that the reader (or listener) takes the author (or speaker) to be committed to the truth of the content.\footnote{We further assume that projectivity is an empirical property of utterance content that can be diagnosed with theoretically untrained native speakers. This position seems to contrast with that of \citet{anand-hacquard2014}, who suggested that examples in which the speaker is understood to be committed to the content of the complement of a `non-factive' predicate merely give rise to the ``illusion of projection'' (p.76).} Projectivity was explored in Exp.~1 with the `certain that' diagnostic for projectivity, which has been used in previous work on projectivity (see, e.g., \citealt{tonhauser-salt26,djaerv-bacovcin-salt27,stevens-etal2017,tbd-variability}).\footnote{For other diagnostics for projectivity see, e.g., \citealt{smith-hall11,xue-onea11} and \citealt{brst-lang11}; see also the discussion in \citealt{tbd-variability}.} On this diagnostic, participants are presented with an utterance in which the content to be investigated occurs in an entailment-canceling environment, like a polar question, as illustrated in (\ref{stim}) for the content of the nominal appositive, that Martha's new car is a BMW.

\begin{exe}

\ex\label{stim} Patrick asks: {\em Was Martha's new car, a BMW, expensive?} 

\end{exe}
To assess whether the relevant content projects, participants are asked whether the speaker is certain of the content; for instance, in (\ref{stim}), participants are asked whether Patrick is certain that Martha's new car is a BMW. 
 

\subsection{Methods}

\paragraph{Participants} 300 participants with U.S.\ IP addresses and at least 99\% of previous HITs approved were recruited on Amazon's Mechanical Turk platform (ages: 20-71, median: 36; 136 female, 157 male, 2 other). They were paid 75 cents for participating in the experiment.

\paragraph{Materials} Exp.~1 explored the projectivity of content of the complement of the 20 clause-embedding predicates in (\ref{pred}). The 5 `factive' predicates in (\ref{pred}a) include the cognitive predicate {\em know}, the change-of-state predicates {\em discover} and {\em reveal}, the sensory predicate {\em see}, and the emotive predicate {\em be annoyed}. The 15 `non-factive' predicates in (\ref{pred}b) are divided into 3 classes, following the discussion in section \ref{s1}. First, the content of the complement of the 4 `plain non-factive' predicates in (\ref{pred}b.i) is typically taken to be neither entailed nor projective. Second, the 2 predicates in  (\ref{pred}b.ii) are `veridical non-factive', i.e., the content of the complement is typically taken to be entailed (see, e.g., \citealt{anand-hacquard2014}) but not projective. Third, the content of the complement of the 9 `projective non-factive' predicates in  (\ref{pred}b.iii)  has been suggested to be projective though not entailed. We included  {\em establish} in this class, despite \citet{wyse} classifying it as `factive' and \citet{swanson2012} classifying it as `veridical non-factive', because of its ability to co-occur with {\em falsely}, as shown in (\ref{establish}) and \citetpos{tbd-variability} observation that the content of its complement is projective. We also included {\em hear} in this class: even though it is often considered a `factive' sensory predicate (e.g., \citealt{beaver-belly,anand-hacquard2014}), it has been observed that {\em hear} also has a `non-factive' reportative evidential sense (see, e.g., \citealt{anderson86,simons07}), especially when it is combined with complements that describe events that cannot be auditorily perceived, as is the case in Exp.~1. 


\begin{exe}
\ex\label{pred} 20 clause-embedding predicates 

\begin{xlist}

\ex `factive' (5): {\em be annoyed, discover, know, reveal, see}

\ex `non-factive' (15)

\begin{xlist}

\ex `plain non-factive' (4): {\em pretend, suggest, say, think}

\ex `veridical non-factive' (2): {\em be right, demonstrate}

\ex `projective non-factive' (9): {\em acknowledge, admit, announce, confess, confirm, establish, hear, inform, prove}


\end{xlist}

\end{xlist}

\end{exe}

The clausal complements of the 20 predicates were realized by 20 clauses (provided in Appendix \ref{a-clauses}), for a total of 400 predicate/clause combinations. These 400 predicate/clause combinations were realized as polar questions by combining them with random proper name subjects, as shown in the sample target stimuli in (\ref{stim-project}), which realize the complement clause {\em Julian dances salsa}. Eventive predicates, like {\em discover} and {\em hear}, were realized in the past tense and stative predicates, like {\em know} and {\em be annoyed}, were realized in the present tense. The indirect object of {\em inform} was realized by the proper name {\em Sam}.  The speaker of the target stimuli was realized by a proper name and displayed in bold-face. The proper names that realized the speakers, the subjects of the clause-embedding predicates and the subjects of the complement clauses were all unique.

\begin{exe}
\ex\label{stim-project} 
\begin{xlist}
\ex {\bf Carol asks:} Did Sandra discover that Julian dances salsa?

\ex {\bf Paul asks:} Does Edward think that Julian dances salsa?
\end{xlist}
\end{exe}

To assess whether participants were attending to the task, the experiment also included the 6  polar question control stimuli shown in (\ref{control}). (The control stimuli were also presented to the participants as utterances by a named speaker.) The content whose projectivity was explored was the main clause content: for instance, in (\ref{control}a), participants were asked whether the speaker was certain that Zack is coming to the meeting tomorrow. We expected participants to judge these main clause contents as not projective.

\begin{exe}
\ex\label{control} 
\begin{xlist}

\ex   Is Zack coming to the meeting tomorrow?

\ex Is Mary's aunt sick?

\ex Did Todd play football in high school?

\ex Is Vanessa good at math?

\ex Did Madison have a baby?

\ex Was Hendrick's car expensive?

\end{xlist}
\end{exe}


Each participant saw a random set of 26 stimuli: each set contained one target stimulus for each of the 20 clause-embedding predicates (each with a unique complement clause) and the same 6 control stimuli. Trial order was randomized.

\paragraph{Procedure} Participants were told to imagine that they are at a party and that, on walking into the kitchen, they overhear somebody ask somebody else a question. Participants were asked to rate whether the speaker was certain of the content of the clausal complement. They gave their responses on a slider marked `no' at one end (coded as 0) and `yes' at the other (coded as 1), as shown in Figure \ref{fig-trial-exp1}.

\begin{figure}[h!]
\begin{center}
\fbox{\includegraphics[width=10cm]{figures/trial-exp1}}
\end{center}
\caption{A sample trial in Experiment 1}\label{fig-trial-exp1}
\end{figure}

After completing the experiment, participants filled out a short, optional survey about their age, their gender, their native language(s) and, if English is their native language, whether they are a speaker of American English (as opposed to, e.g., Australian or Indian English). To encourage them to respond truthfully, participants were told that they would be paid no matter what answers they gave in the survey.

\paragraph{Data exclusion}
Prior to analysis, the data from 13 participants who did not self-identify as native speakers of American English were excluded. For the remaining 287 participants, we inspected their responses to the 6 control stimuli, for which we expected low responses because main clause content does not project. As expected, the group mean on these control stimuli was low (.14), but the response means of 4 participants were more than 3 standard deviations above the group mean. Closer inspection revealed that these participants' responses to the control stimuli were systematically higher, suggesting that these participants did not attend to the task or interpreted the task differently. The data from these 4 participants were also excluded, resulting in a group mean of .13 on the control stimuli and leaving data from 283 participants (ages 20-71; median: 36; 127 female, 150 male, 2 other).

\subsection{Results and discussion}

Each predicate received 283 certainty ratings and each predicate/clause combination received between 5 and 27 ratings (mean: 14.2). The box plot in Figure \ref{f-projectivity} shows the certainty ratings for the target stimuli by predicate, collapsing over complement clauses, as well as the certainty ratings of the main clause control stimuli (abbreviated `MC'). `Factive' predicates are given in darker blue, `veridical non-factive' predicates in lighter blue, `plain non-factive' predicates in brown and `projective non-factive' predicates in black. (This color scheme is used throughout the remainder of the paper.) The mean certainty ratings are given by grey dots and the notches indicate medians. 
%These certainty ratings further support \citetpos{tbd-variability} claim that projectivity is a gradient property of utterance content: the mean certainty ratings of the 21 expressions spread from floor to ceiling, and for many contents there are (non-outlier) certainty ratings from floor to ceiling. 
The mean certainty ratings are consistent with impressionistic judgments reported in the literature. First, the ratings for main clause content are lowest overall, as expected for non-projective content. Second, the ratings for `factive' predicates are among the highest overall, suggesting comparatively high projectivity of the content of the complement. Third, the ratings for `plain non-factive' predicates are among the lowest overall, suggesting comparatively low projectivity of the content of the complement. Finally, the mean certainty ratings of many of the `projective non-factives' are higher than those of main clauses as well as of `plain non-factives', which confirms the intuitions of authors like \citet{schlenker10,anand-hacquard2014} and \citet{spector-egre2015} that the content of the complement of some `non-factive' predicates may project.

\begin{figure}[H]
\centering

\includegraphics[width=.75\paperwidth]{../results/5-projectivity-no-fact/graphs/boxplot-projectivity}

\caption{Box plot of certainty ratings by predicate, collapsing over complement clauses. Grey dots indicate means and notches indicate medians. `MC' abbreviates main clause controls. `Factive' predicates are given in darker blue, `veridical non-factive' ones in lighter blue, `plain non-factive' ones in brown and `projective non-factive' ones in black.}
\label{f-projectivity}
\end{figure}

Recall from section \ref{s1} that the distinction between `factive' and `non-factive' predicates relies on two assumptions: i) content is either projective or not, and ii) the content of the complement of `factive' predicates is projective whereas that of `non-factive' predicates, with the exception of `projective non-factive' predicates, is not.  Neither of these assumptions is empirically supported by the findings of Exp.~1. Regarding the first assumption, the certainty ratings plotted in Figure \ref{f-projectivity} provide further support for  \citetpos{tbd-variability} finding that projectivity is a gradient property of utterance content, not a binary categorical one.  With respect to by-predicate projection variability, we observe, for instance, that the mean certainty ratings for the contents of the complements of the 20 predicates are spread out from .17 for {\em pretend} to .86 for {\em be annoyed}, suggesting gradience in average projectivity. We also observe gradient within-predicate projection variability: for many predicates, the certainty ratings range from floor to ceiling, suggesting gradience in the extent to which participants took speakers to be committed to the content of the complement of particular clause-embedding predicates. For the `factive' predicate {\em reveal}, for instance, Figure \ref{f-projectivity} shows that the middle 50\% of certainty ratings fell in the range from .5 to .95. In sum, projectivity is a gradient property of utterance content, not a binary categorical one.

%14 reveal     0.688 0.0362  0.0377 0.652
%17 discover   0.763 0.0328  0.0317 0.730
%19 see        0.795 0.0314  0.0300 0.763
%20 know       0.841 0.0311  0.0263 0.810
%21 be_annoyed 0.859 0.0276  0.0263 0.831

As discussed in section \ref{s1}, the finding that projectivity is a gradient property of utterance content gives rise to the question of whether projectivity distinguishes `factive' predicates from `non-factive' ones (modulo `projective non-factive' ones) and how projective the content of the complement of a clause-embedding predicate needs to be in order for the predicate to be considered `factive'. To determine which clause-embedding predicates differed from each other in the projectivity of the content of the  complement, we conducted post hoc pairwise comparisons using Tukey's method (allowing for by-participant variability), using the \verb|lsmeans| package \citep{tukey} in R \citep{r}. We included the main clause controls in the analysis to identify predicates for which the content of the complement was not projective (7358 data points total). The $p$-values for each comparison pair are shown in Table \ref{t-pairwise-proj}. 

\begin{table}[h!]
\setlength\tabcolsep{3pt}

\centering
\small

\begin{tabular}{l l l l l l l l l l l l l l l l l l l l l }
\toprule

& \rot{MC} &   \rot{\color{brown}{\em pretend}\color{black}} & \rot{\color{airforceblue}{\em be right}\color{black}} & \rot{\color{brown}{\em think}\color{black}} &  \rot{\color{brown}{\em suggest}\color{black}} &  \rot{\color{brown}{\em say}\color{black}} & \rot{\color{black}{\em prove}\color{black}} & \rot{{\em confirm}} & \rot{\color{black}{\em establish}\color{black}} & \rot{\color{airforceblue}{\em demonstrate}\color{black}} & \rot{{\em announce}} & \rot{{\em confess}} & \rot{\color{black}{\em admit}\color{black}} & \rot{\color{blue}{\em reveal}\color{black}}  & \rot{{\em acknowledge}}  & \rot{\color{black}{\em hear}\color{black}}  & \rot{\color{blue}{\em discover}\color{black}} & \rot{\color{black}{\em inform}\color{black}}  & \rot{\color{blue}{\em see}\color{black}}  & \rot{\color{blue}{\em know}\color{black}}  \\

\midrule

\color{brown}{\em pretend}\color{black}	&  n.s. & - & - & - & - & - & - & - & - & - & - & - & - & - & - & - & - & - & - \\
\color{airforceblue}{\em be right}\color{black}	& ** & n.s. & - & - & - & - & - & - & - & - & - & - & - & - & - & - & - & - & - & - \\
\color{brown}{\em think}\color{black}		& ** & n.s. & n.s. & - & - & - & - & - & - & - & - & - & - & - & - & - & - & - & - & - \\
\color{brown}{\em suggest}\color{black}	& ***	& . & n.s. & n.s. & - & - & - & - & - & - & - & - & - & - & - & - & - & - & - & - \\
\color{brown}{\em say}	\color{black}	& ***	& ** & n.s. & n.s. & n.s. & - & - & - & - & - & - & - & - & - & - & - & - & - & - & - \\
\color{black}{\em prove}\color{black}		& ***	& *** &  ***  & ** & . & n.s. & - & - & - & - & - & - & - & - & - & - & - & - & - & - \\
\color{black}{\em confirm}\color{black}	&***	& ***  &  ***  &  ***  &  ***  & ** & n.s. & - & - & - & - & - & - & - & - & - & - & - & - & - \\
\color{black}{\em establish}\color{black}	&***	&  ***  &  ***  &  ***  &  ***  &  ***  & n.s & n.s. & - & - & - & - & - & - & - & - & - & - & - & - \\
\color{airforceblue}{\em demonstrate}\color{black} &*** & *** & *** & *** &  *** &  *** &  ***  &  ***  &  ***  & - & - & - & - & - & - & - & - & - & - & - \\
\color{black}{\em announce}\color{black}		& ***	& *** & *** & *** & *** & *** & *** &  ***  &  ***  & ** & - & - & - & - & - & - & - & - & - & - \\
\color{black}{\em confess}\color{black}	& ***	& *** & *** & *** & *** & *** & *** & *** & *** &  ***  & n.s. & - & - & - & - & - & - & - & - & - \\
\color{black}{\em admit}\color{black}		& ***	& *** & *** & *** & *** & *** & *** & *** & *** &  ***  & . & n.s. &  - & - & - & - & - & - & - & - \\
\color{blue}{\em reveal}\color{black}			& ***	& *** & *** & *** & *** & *** & *** & *** & *** &  ***  &  ***  & n.s. & n.s. & - & - & - & - & - & - & - \\
\color{black}{\em acknowledge}\color{black}	& ***	& *** & *** & *** & *** & *** & *** & *** & *** &  ***  &  ***  & ** & n.s. & n.s. & - & - & - & - & - & - \\
\color{black}{\em hear}\color{black}		& ***	& *** & *** & *** & *** & *** & *** & *** & *** & *** &  ***  &  ***  & ** & n.s. & n.s. & - & - & - & - & - \\
\color{blue}{\em discover}\color{black}	& ***		& *** & *** & *** & *** & *** & *** & *** & *** & *** &  ***  &  ***  &  ***  & * & n.s. & n.s. & - & - & - & - \\
\color{black}{\em inform}\color{black}		&***		& *** & *** & *** & *** & *** & *** & *** & *** & *** & *** & *** &  ***  & ** & * & n.s. & n.s. & - & - & - \\
\color{blue}{\em see}\color{black}		&***		& *** & *** & *** & *** & *** & *** & *** & *** & *** & *** &  ***  &  ***  &  ***  & ** & n.s. & n.s. & n.s. & - & - \\
\color{blue}{\em know}\color{black}		&***		& *** & *** & *** & *** & *** & *** & *** & *** & *** & *** & *** & *** & *** & *** & *** & * & n.s. & n.s. & -  \\
\color{blue}{\em be annoyed}\color{black}	&***		& *** & *** & *** & *** & *** & *** & *** & *** & ***  & ***  & *** & *** & *** & *** & *** & ** & . & n.s. & ns  \\

\bottomrule
\end{tabular}
\caption{P-values associated with pairwise comparison of certainty ratings of the 20 clause-embedding predicates and the main clause controls using Tukey's method. `***' indicates significance at .0001, `**' at .01, `*' at .05, `.' marginal significance at .1, and `n.s' indicates no significant difference in means. `MC' abbreviates main clause controls. `Factive' predicates are given in darker blue, `veridical non-factive' ones in lighter blue, `plain non-factive' ones in brown and `projective non-factive' ones in black.}\label{t-pairwise-proj}
\end{table} 

The analysis revealed that, for the most part, the content of the complement of `plain non-factive' and `veridical non-factive' predicates is less projective than that of `factive' and `projective non-factive' predicates. Critically, however, the observed by-predicate differences do not align with the assumed categorical distinction between, on the one hand, `factive' and `projective non-factive' predicates and, on the other hand, `plain non-factive' and `veridical non-factive' predicates. First, there are pairs of predicates that do not pattern as expected: the content of the complement of the `veridical non-factive' predicate {\em demonstrate} is more projective than that of the `projective non-factive' predicates {\em establish, confirm} and {\em prove}, and the content of the complement of the `projective non-factive' predicate {\em prove} is indistinguishable in its projectivity from that of the `plain non-factive' predicates {\em say} and {\em suggest}. Second, because of the observed by-predicate variability, `veridical non-factive' and `plain non-factive' predicates do not form a natural class with respect to the projectivity of the content of the complement, and neither do `factive' and `projective non-factive' predicates. That is, for the 6 `veridical non-factive' and `plain non-factive' predicates, the content of the complement of {\em demonstrate} is more projective than that of {\em be right} and of the 4 `plain non-factive' predicates. Likewise, for the 14 `factive' and `projective non-factive' predicates, there is widespread by-predicate variability: for instance, the content of the complement of {\em be annoyed} and {\em know} is more projective than that of {\em discover} and {\em hear}, that of {\em inform} is more projective than that of {\em acknowledge, reveal} and {\em admit}, that of {\em discover} is more projective than that of {\em reveal, admit} and {\em confess}, and that of {\em confess} is more projective than that of {\em establish}.\footnote{These findings align with those of \citealt{tbd-variability}, who also observed that the content of the complement of {\em reveal} was less projective than that of {\em discover}, which in turn was less projective than that of {\em know} and {\em be annoyed}, and that the content of {\em establish} was more projective than that of {\em confess}. There are two differences between the findings of Exp.~1 and of \citetpos{tbd-variability} Exps.~1a and 1b. First, the projectivity of the content of the complement of {\em reveal} was significantly higher than that of {\em confess} in their Exp.~1b, but it is only numerically higher in Exp.~1. Second, the certainty ratings were overall higher in \citetpos{tbd-variability} Exps.~1 than in our Exp.~1. For instance, the content of the complement of {\em be annoyed}, which was the most projective in both sets of experiments, received mean certainty ratings of .96  and .92 in their Exps.~1a and 1b, respectively, but only a .86 in our Exp.~1. Similarly, the content of the complement of {\em discover} received mean certainty ratings of .86 and .85 in their Exps.~1a and 1b, respectively, but only a .76 in our Exp.~1. Both of these differences may be due to the fact that \citet{tbd-variability} primarily explored highly projective content whereas our Exp.~1 also included a wide range of less projective content: it is possible that participants' certainty ratings were influenced by the overall projectivity of the contents explored. We leave this matter to future research.} Thus, the findings of Exp.~1 do not support the assumed binary categorical distinction between, on the one hand, `factive' and `projective non-factive' predicates, and, on the other hand, `plain non-factive' and `veridical non-factive' predicates.

Do the findings of Exp.~1 support an alternative binary categorical distinction between the 20 predicates with respect to the projectivity of the content of the complement? We argue that this is not the case. To illustrate, we note that the pairwise comparison reveals that only the content of the complement of {\em pretend} is indistinguishable in projectivity from main clause content. This finding suggests that only this content is not projective and that the content of the complement of the remaining 19 predicates, including that of the other 3 `plain non-factive' and the 2 `veridical non-factive' predicates, can be considered at least weakly projective.\footnote{\citet{spector-egre2015} suggested that it is not clear whether the content of the complement of {\em say} is projective (p.1739) but that it is definitely less projective than that of {\em tell}. The findings of Exp.~1 suggest that it is very weakly projective.} Now consider an alternative binary categorical division among the 20 clause-embedding predicates: the first set consists of {\em pretend, be right, think, suggest} and {\em say}, i.e., the 5 predicates for which the content of the complement is least projective, and the second set of the remaining 15 predicates. Neither of the two sets of predicates forms a natural class: the first set includes predicates for which the content of the complement is not projective as well as predicates for which it is weakly projective, and the second set includes predicates that exhibit significant differences in the projectivity of the content of the complement. Furthermore, the projectivity of the content of the complement of {\em say} and {\em suggest} (first set) is indistinguishable from that of {\em prove} (second set), which illustrates the arbitrariness of the dividing line. More generally, we argue that any binary categorical distinction between the 20 predicates is bound to be arbitrary because of the rampant by-predicate variability and the observation that the content of the complement of most of these 20 predicates is at least weakly projective.

In sum, Exp.~1 confirmed \citetpos{tbd-variability} finding that projectivity is a gradient property of utterance content and showed that projectivity does not categorically distinguish the content of the complement of `factive' (and `projective non-factive') predicates from that of `(plain and veridical) non-factive' predicates. The experiments described in the next section were designed to explore whether the content of the complement of the same 20 clause-embedding predicates is entailed, to identify whether entailment and projectivity jointly distinguish `factive' and `non-factive' predicates, as per the definition in (\ref{def}).


%Exp 1b: discover $<$ reveal $<$ confess $<$ establish

% Exp 1: discover $<$ reveal, confess $<$ establish, but reveal is not more projective than confess

% 1 MC         0.133 0.00978 0.0100 0.123
% 9 establish  0.365 0.0348  0.0368 0.330
%12 confess    0.626 0.0396  0.0354 0.587
%14 reveal     0.688 0.0362  0.0377 0.652
%17 discover   0.763 0.0328  0.0317 0.730
%20 know       0.841 0.0311  0.0263 0.810
%21 be_annoyed 0.859 0.0276  0.0263 0.831

% Exp 1a
% annoyed       0.9595714 0.01473729 1675.74 0.9306660 0.9884769
% discover      0.8556667 0.01473729 1675.74 0.8267612 0.8845721
% know          0.9233810 0.01473729 1675.74 0.8944755 0.9522864

% Exp 1b
% confessed     0.6869787 0.01499147 1739.4 0.6575755 0.7163819
% discovered    0.8534043 0.01499147 1739.4 0.8240011 0.8828075
% established   0.4177447 0.01499147 1739.4 0.3883415 0.4471479
% is_annoyed    0.9228936 0.01499147 1739.4 0.8934904 0.9522968
% revealed      0.7784255 0.01499147 1739.4 0.7490223 0.8078287
% saw           0.8905957 0.01499147 1739.4 0.8611925 0.9199989



%  verb        Mean   CILow CIHigh  YMin
%   <fct>      <dbl>   <dbl>  <dbl> <dbl>
% 1 MC         0.133 0.00978 0.0100 0.123
% 2 pretend    0.166 0.0238  0.0255 0.142
% 3 be_right   0.199 0.0282  0.0305 0.170
% 4 think      0.205 0.0232  0.0249 0.182
% 5 suggest    0.238 0.0285  0.0285 0.209
% 6 say        0.254 0.0302  0.0324 0.224
% 7 prove      0.308 0.0293  0.0289 0.279
% 8 confirm    0.350 0.0321  0.0335 0.318
% 9 establish  0.365 0.0348  0.0368 0.330
%10 demonstra? 0.485 0.0384  0.0369 0.446
%11 announce   0.575 0.0361  0.0375 0.539
%12 confess    0.626 0.0396  0.0354 0.587
%13 admit      0.647 0.0376  0.0347 0.610
%14 reveal     0.688 0.0362  0.0377 0.652
%15 acknowled? 0.712 0.0341  0.0328 0.678
%16 hear       0.733 0.0349  0.0340 0.698
%17 discover   0.763 0.0328  0.0317 0.730
%18 inform     0.789 0.0347  0.0289 0.754
%19 see        0.795 0.0314  0.0300 0.763
%20 know       0.841 0.0311  0.0263 0.810
%21 be_annoyed 0.859 0.0276  0.0263 0.831

%  verb        Median Quantile1 Quantile2 Quantile4 Quantile5
% 1 acknowledge  0.81       1        0.97      0.51          0
% 2 admit        0.73       1        0.92      0.455         0
% 3 announce     0.580      1        0.855     0.34          0
% 4 be_annoyed   0.95       1        0.99      0.85          0
% 5 be_right     0.11       1        0.305     0.02          0
% 6 confess      0.71       1        0.91      0.42          0
% 7 confirm      0.290      1        0.545     0.09          0
% 8 MC           0.03       1        0.16      0             0
% 9 demonstrate  0.49       1        0.77      0.21          0
%10 discover     0.87       1        0.98      0.64          0
%11 establish    0.3        1        0.52      0.08          0
%12 hear         0.85       1        0.97      0.6           0
%13 inform       0.9        1        0.98      0.7           0
%14 know         0.95       1        0.99      0.82          0
%15 pretend      0.08       0.98     0.22      0.01          0
%16 prove        0.25       1        0.48      0.07          0
%17 reveal       0.8        1        0.95      0.5           0
%18 say          0.19       1        0.425     0.03          0
%19 see          0.91       1        0.98      0.72          0
%20 suggest      0.16       1        0.37      0.04          0
%21 think        0.15       0.97     0.31      0.03          0
 
\section{Experiment 2: Entailment}\label{s3}

The two experiments described in this section were designed explore whether the content of the complement is entailed by the content of unembedded matrix sentences with the 20 clause-embedding predicates. We assume a standard definition of entailment, according to which entailment is a binary, categorical relation between two contents: given sentences $\phi$ and $\psi$, the content of $\phi$ entails the content of $\psi$ if and only if every world in which the content of $\phi$ is true is also a world in which the content of $\psi$ is true. Thus, if there is even one world in which $\phi$ is true but $\phi$ is false, then the content of $\phi$ does not entail the content of $\psi$. 

Given this standard definition of the entailment relation, it is straightforward to establish that the contents of two natural language sentences $\phi$ and $\psi$ do not stand in an entailment relation, namely by identifying a world in which the content of $\phi$ is true and the content of $\psi$ is false. In this paper, however, we are also interested in establishing that the contents of two sentences $\phi$ and $\psi$  do stand in an entailment relation. Establishing such a relation is relatively straightforward if the content of some expression in $\phi$ denotes a subset of the content of some expression in $\psi$. For instance, in the example in (\ref{ent1}a),  the content of $\phi$ entails the content of $\psi$ because the denotation of {\em black cat} in $\phi$ is a subset of the denotation of {\em cat} in $\psi$. No such structural relation exists, however, between the content of a sentence with a clause-embedding predicate and the content of the complement, as in the example in (\ref{ent1}b). Furthermore, it is impossible to empirically verify that the content of $\phi$ entails the content of $\psi$ in (\ref{ent1}b) because it is impossible to verify that $\psi$ is true in all worlds in which $\phi$ is true. 

\begin{exe}
\ex\label{ent1}
\begin{xlist}
\ex $\phi$: Sam owns a black cat. \hspace*{1.5cm} $\psi$: Sam owns a cat.

\ex $\phi$: Sam knows that it's raining. \hspace*{.6cm} $\psi$: It's raining.

\end{xlist}
\end{exe}

Thus, to establish whether the content of an unembedded matrix sentence with a clause-embedding predicate entails the content of the clausal complement, we resort to diagnostics that are taken to provide evidence for an entailment relation. Two standard diagnostics are given in (\ref{diag}); these were already illustrated with the emotive predicate {\em regret} in section \ref{s1}. According to the `inference diagnostic' in (\ref{diag}a), the content of an unembedded matrix sentence with a clause-embedding predicate entails the content of the complement if and only if the content of the complement definitely follows from a true statement of the unembedded matrix sentence. According to the `contradictoriness diagnostic' in (\ref{diag}b), such an entailment relationship holds if and only an utterance of the unembedded matrix sentence that is followed by a denial of the content of the complement is contradictory. 

\begin{exe}
\ex\label{diag} Two diagnostics for entailment \hfill (see, e.g., \citealt[\S3.1]{ccmg90})
\begin{xlist}
\ex  Inference diagnostic \\ $\phi$ entails $\psi$ if and only if, if $\phi$ is true, then (the truth of) $\psi$ definitely follows. 

\ex  Contradictoriness diagnostic \\ $\phi$ entails $\psi$ if and only if sentences of the form {\em $\phi$ and not $\psi$} are contradictory. 

\end{xlist}
\end{exe}
These two diagnostics for entailment are applied to the content of the complement of the 20 clause-embedding predicates in Exps.~2a and 2b, respectively. Both experiments include control stimuli for which the relevant inference definitely follows (Exp.~2a) and that are definitely contradictory (Exp.~2b). We assume that the content of the clausal complement of sentences with a clause-embedding predicate is not an entailment if responses to the content of the clausal complement are significantly different from responses to the control stimuli. According to the diagnostics, identifying contents of clausal complements that are entailments, however, requires arguing from a null result: the content of the complement of sentences with a clause-embedding predicate is an entailment if responses to the content of the complement are not significantly different from responses to the control stimuli. We present the findings of the two experiments in sections \ref{s31} and \ref{s32}, and then discuss the findings jointly in section \ref{s33}.

\subsection{Experiment 2a: Inference diagnostic for entailment}\label{s31}

This experiment explored whether the content of the clausal complement of the 20 clause-embedding predicates is an entailment based on the inference diagnostic for entailment in (\ref{diag}a). In contrast to the entailment relationship, which is binary and categorical, the inference relationship between the contents of two sentences $\phi$ and $\psi$ is gradient: given a true statement $\phi$, the truth of $\psi$ follows to varying degrees. To illustrate, assume that Chris wears his pajamas whenever he is at home and that he doesn't wear pajamas outside of his home. Now consider the contents of the sentences $\phi_1$ to $\phi_5$ in (\ref{chris}). Given the aforementioned assumption, the (truth of the) content of the statement $\psi$ {\em Chris is wearing his pajamas} definitely follows from a true statement of $\phi_1$, it follows to increasingly lower degrees from true statements of $\phi_2$ to $\phi_4$ and it definitely does not follow from a true statement of $\phi_5$. 

\begin{exe}
\ex\label{chris}
\begin{xlist}
\exi{$\phi_1$} Chris is at home.
\exi{$\phi_2$} Chris is likely at home.
\exi{$\phi_3$} Chris is possibly at home.
\exi{$\phi_4$} There is a slight possibility that Chris is at home.
\exi{$\phi_5$} Chris is not at home.
\end{xlist}
\end{exe}

In applying the inference diagnostic for entailment in Exp.~2a, we assume that if the content of $\phi$ entails the content of $\psi$, then the (truth of the) content of $\psi$ definitely follows from true statements of $\phi$, i.e., the strength of the inference is at ceiling. If the content of $\phi$ does not entail the content of $\psi$, then the strength of the inference is somewhere below ceiling. Exp.~2a investigated the strength of the inference from true statements of unembedded matrix sentences with the 20 clause-embedding predicates to the contents of the complements by collecting gradient inference ratings.

\subsubsection{Methods}

\paragraph{Participants} 300 participants with U.S.\ IP addresses and at least 99\% of previous HITs approved were recruited on Amazon's Mechanical Turk platform (ages: 19-69, median: 36; 152 female, 148 male). They were paid 75 cents for participating in the experiment.

\paragraph{Materials} The target stimuli were 400 unembedded matrix sentences that were constructed by combining the 400 predicate/clause combinations of Exp.~1 with a random proper name subject whose gender differed from the gender of the proper name subject of the embedded clause. As illustrated in the sample stimuli in (\ref{stims2}), these matrix sentences were presented to participants as true statements. For each of the 400 target stimuli, the inference that was tested was the inference to the content of the clausal complement: for instance, participants were asked about (\ref{stims2}a) whether it follows that Danny ate the last cupcake and about (\ref{stims2}b) whether it follows that Emma studied on Saturday morning.

\begin{exe}
\ex\label{stims2}
\begin{xlist}
\ex {\bf What is true:} Melissa knows that Danny ate the last cupcake.
%\\ Does it follow that Danny ate the last cupcake?
\ex {\bf What is true:} Jerry pretended that Emma studied on Saturday morning.
%\\ Does it follow that Emma studied on Saturday morning?
\end{xlist}
\end{exe}

The experiment also included eight control stimuli that were used to assess whether participants were attending to the task, whether they interpreted the task correctly, as well as to identify inference ratings that were at ceiling and at floor. For the four entailing control stimuli in (\ref{control-good2}), the tested inferences are lexical entailment of the true statements: for instance, that Frederick solved the problem is a lexical entailment of the true statement in (\ref{control-good2}a) that Frederick managed to solve the problem and that the window broke is a lexical entailment of the true statement in (\ref{control-good2}b) that Tara broke the window with a bat. We expected the inference ratings for these lexical entailments to be at ceiling, i.e., to be entailments. In contrast, the inferences tested with the four non-entailing control stimuli in (\ref{control-bad2}) definitely do not follow from these true statements: in (\ref{control-bad2}a), that Dana wears a wig does not follow from Dana watching a movie last night, and that Madison closed the window is the negation of what actually follows from the true statement in (\ref{control-bad2}c). We expected the inference ratings for these four control stimuli to be at floor.

\begin{exe}
\ex\label{control-good2} Entailing control stimuli
\begin{xlist}
\ex {\bf What is true:} Zack bought himself a car this morning. (Tested inference: Zack owns a car.)
\ex {\bf What is true:} Tara broke the window with a bat. (Tested inference: The window broke.)
\ex {\bf What is true:} Frederick managed to solve the problem. (Tested inference: Frederick solved the problem.)
\ex {\bf What is true:} Vanessa happened to look into the mirror. (Tested inference: Vanessa looked into the mirror.)
\end{xlist}
\ex\label{control-bad2} Non-entailing control stimuli
\begin{xlist}
\ex {\bf What is true:} Dana watched a movie last night. (Tested inference: Dana wears a wig.)
\ex {\bf What is true:} Hendrick is renting an apartment. (Tested inference: The apartment has a balcony.)
\ex {\bf What is true:} Madison was unsuccessful in closing the window. (Tested inference:  Madison closed the window.)
\ex {\bf What is true:} Sebastian failed the exam. (Tested inference: Sebastian did really well on the exam.)
\end{xlist}
\end{exe}

Each participant saw a random set of 28 stimuli: each set contained one target stimulus for each of the 20 predicates (each with a unique complement clause) and the same 8 control stimuli. Trial order was randomized.

\paragraph{Procedure} Participants were told that they would read true statements and that they would be asked to assess whether a second statement follows from the first, true statement. On each trial, participants read the true statement and were asked to respond to the question {\em Does it follow that\ldots} where the continuation was realized by the complement of the clause-embedding predicate. Participants gave their responses on a sliding scale from `definitely doesn't follow' (coded as 0) to `definitely follows' (coded as 1), as shown in Figure \ref{f-trial-exp3}.

\begin{figure}[H]
\begin{center}
\fbox{\includegraphics[width=13cm]{figures/inference-trial}}
\end{center}
\caption{A sample trial in Experiment 2a}\label{f-trial-exp3}
\end{figure}

To familiarize participants with the task, they first responded to the two familiarization stimuli in (\ref{train2}), where the inference tested was the inference to the content of the clausal complement. Participants who rated (\ref{train2}a) in the lower half of the sliding scale or (\ref{train2}b) in the upper half of the sliding scale were given an explanation for why their answer was wrong. Participants could only advance to the 28 stimuli if they gave a plausible rating to the two familiarization stimuli, i.e., a rating in the upper half of the sliding scale for (\ref{train2}a) and in the lower half for (\ref{train2}b).

\begin{exe}
\ex\label{train2}
\begin{xlist}
\ex {\bf What is true:} Drew is correct that Patty lives in Canada. 

\ex {\bf What is true}: Drew believes that Patty lives in Canada.
\end{xlist}
\end{exe}

After responding to the 28 stimuli, participants filled out a short, optional survey about their age, their gender, their native language(s) and, if English is their native language, whether they are a speaker of American English (as opposed to, e.g., Australian or Indian English). To encourage them to respond truthfully, participants were told that they would be paid no matter what answers they gave in the survey.

\paragraph{Data exclusion}

Prior to analysis, the data from 14 participants who did not self-identify as native speakers of American English were excluded. For the remaining 286 participants, we inspected their responses to the 8 control stimuli. The group means for the four control stimuli in (\ref{control-good2}), for which the strength of the inference was expected to be at ceiling, was .94. The group mean for the four control stimuli in (\ref{control-bad2}), for which the strength of the inference was expected to be at floor, was .05. These group means suggest that, overall, participants attended to and understood the task. The response means of 7 participants were more than 3 standard deviations below the group mean for the control stimuli in (\ref{control-good2}) or above the group mean for the control stimuli in (\ref{control-bad2}). Closer inspection revealed that these participants' responses to the control stimuli were systematically higher or lower, respectively, suggesting that these participants did not attend to the task or interpreted the task differently. The data from these 7 participants were also excluded, leaving data from 279 participants (ages 19-69; median: 36; 142 female, 137 male) and resulting in new group means of .95 for the control stimuli in (\ref{control-good2}) and .04 for the control stimuli in (\ref{control-bad2}).

\subsubsection{Results}

Each of the 20 clause-embedding predicates received 279 ratings and each of the 400 predicate/clause combinations received between 5 and 24 ratings (mean: 13.9). The box plot in Figure \ref{f-veridicality-predicate2} shows the inference ratings by predicate, collapsing over the 20 complement clauses, as well as the inference ratings of the non-entailing controls in (\ref{control-bad2}) and the entailing controls in (\ref{control-good2}). The `plain non-factive' predicates {\em pretend, think, suggest} and {\em say} received among the lowest ratings overall, as expected from the assumption that the content of the complement of these predicates is not entailed.  The mean inference ratings of several predicates were at or close to ceiling (recall the group mean of .95 for the entailing control stimuli in (\ref{control-good2})), including the `factive' predicates {\em see} (.94), {\em discover} (.94),  {\em know} (.93) and {\em be annoyed} (.92), as well as the `veridical non-factive' predicate {\em be right} (.95), but also the `projective non-factive' predicates {\em prove} (.95),  {\em establish} (.90), {\em admit} (.90), and {\em acknowledge} (.90). This finding suggests that participants took the truth of the content of the clausal complement of these predicates to (mostly) follow from the true statements of unembedded matrix sentences with these predicates. However, for the `factive' predicate {\em reveal}, the larger box and longer lower whisker already suggest that there were participants for whom the content of the clausal complement of this predicate did not follow as strongly from true statements with this predicate. The same holds for the purported `veridical non-factive' predicate {\em demonstrate}, whose mean inference rating was not at ceiling (.84). We also note that the inference ratings for the `projective non-factive' predicate {\em hear} were quite low, which is compatible with the assumption, discussed in section \ref{s2}, that participants  interpret the target stimuli with {\em hear} using the evidential sense of {\em hear}, not the sensory sense of {\em hear}. 

%          verb      Mean       YMin      YMax
%1  acknowledge 0.8980287 0.87993817 0.9159525
%2        admit 0.8984946 0.88035663 0.9165950
%3     announce 0.8034409 0.77417742 0.8306201
%4   be_annoyed 0.9150896 0.89709409 0.9314023
%5     be_right 0.9472043 0.93537455 0.9581362
%6      confess 0.8837993 0.86551523 0.9022661
%7      confirm 0.9361290 0.92328853 0.9485726
%8  demonstrate 0.8428674 0.81867204 0.8683889
%9     discover 0.9394624 0.92898835 0.9494991
%10   establish 0.8989247 0.88118100 0.9168486
%11        hear 0.4996057 0.46280108 0.5363100
%12      inform 0.8301434 0.80379301 0.8570609
%13        know 0.9251613 0.91092563 0.9387849
%14     pretend 0.1219713 0.09496774 0.1505968
%15       prove 0.9492832 0.93799194 0.9594292
%16      reveal 0.8963082 0.87770520 0.9145878
%17         say 0.6786022 0.64214785 0.7137796
%18         see 0.9426523 0.93217921 0.9526900
%19     suggest 0.3413262 0.30690950 0.3740600
%20       think 0.3138351 0.28154032 0.3453790

\begin{figure}[h!]
\centering

\includegraphics[width=.75\paperwidth]{../results/4-veridicality3/graphs/boxplot-inference}

\caption{Box plot of inference ratings by predicate, collapsing across complement clauses, including the non-entailing controls in (\ref{control-bad2}) (abbreviated `non-ent.\ C') and the entailing controls in (\ref{control-good2}) (abbreviated `entailing C'). Grey dots indicate means and notches indicate medians. `Factive' predicates are given in darker blue, `veridical non-factive' ones in lighter blue, `plain non-factive' ones in brown and `projective non-factive' ones in black.}
\label{f-veridicality-predicate2}
\end{figure}

To determine which clause-embedding predicates are such that the content of the complement is entailed, i.e., for which clause-embedding predicates the strength of the inference to the content of the complement is at ceiling, we conducted post hoc pairwise comparisons of the predicates and the entailing control stimuli in (\ref{control-good2}) using Tukey's method (allowing for by-participant variability), using the \verb|lsmeans| package \citep{tukey} in R \citep{r}  (6696 data points total). The $p$-values for each comparison pair are shown in Table  \ref{t-pairwise2}. These results suggest no difference in the strength of the inference to the content of the entailing controls and the content of the complement of the `factive' predicates {\em see, discover, know} and {\em be annoyed}, of the `veridical non-factive' predicate {\em be right} and the `projective non-factive' predicates {\em prove} and {\em confirm}. These findings suggest that the content of the complement of these predicates is entailed.

\begin{table}[h!]
\setlength\tabcolsep{3pt}

\centering
\small

\begin{tabular}{l l l l l l l l l l l l l l l l l l l l l }
\toprule

 &  \rot{\color{brown}{\em pretend}\color{black}} & \rot{\color{brown}{\em think}\color{black}} & \rot{\color{brown}{\em suggest}\color{black}} &  \rot{\color{black}{\em hear}\color{black}} &  \rot{\color{brown}{\em say}\color{black}} & \rot{{\em announce}} & \rot{{\em inform}} & \rot{\color{airforceblue}{\em demonstrate}\color{black}} & \rot{\color{black}{\em confess}\color{black}} & \rot{\color{blue}{\em reveal}\color{black}} & \rot{{\em acknowledge}} & \rot{{\em admit}} & \rot{\color{black}{\em establish}\color{black}}  & \rot{\color{blue}{\em be annoyed}\color{black}}  & \rot{\color{blue}{\em know}\color{black}}  & \rot{\color{black}{\em confirm}\color{black}} & \rot{\color{blue}{\em discover}\color{black}}  & \rot{\color{blue}{\em see}\color{black}}  & \rot{\color{airforceblue}{\em be right}\color{black}} & \rot{\color{black}{\em prove} \color{black}}  \\
\midrule

\color{brown}{\em think}\color{black}		& *** & - & - & - & - & - & - & - & - & - & - & - & - & - & - & - & - & - & - & -\\
\color{brown}{\em suggest}\color{black}			& *** & n.s. & - & - & - & - & - & - & - & - & - & - & - & - & - & - & - & - & - & - \\
\color{black}{\em hear}	\color{black}		& *** & *** & *** & - & - & - & - & - & - & - & - & - & - & - & - & - & - & - & - & -\\
\color{brown}{\em say}\color{black}		& *** & *** & *** & *** & - & - & - & - & - & - & - & - & - & - & - & - & - & - & - & -\\
\color{black}{\em announce}\color{black}		& *** & *** & *** & *** & *** & - & - & - & - & - & - & - & - & - & - & - & - & - & - & -\\
\color{black}{\em inform}\color{black}		& *** & *** & *** & *** & *** & n.s. & - & - & - & - & - & - & - & - & - & - & - & - & - & -\\
\color{airforceblue}{\em demonstrate}\color{black}		& *** & *** & *** & *** & *** & n.s. & n.s. & - & - & - & - & - & - & - & - & - & - & - & - & -\\
\color{black}{\em confess}\color{black}		& *** & *** & *** & *** & *** & *** & * & n.s. & - & - & - & - & - & - & - & - & - & - & - & -\\
\color{blue}{\em reveal}\color{black}		& *** & *** & *** & *** & *** & *** & ** & * & n.s. & - & - & - & - & - & - & - & - & - & - & -\\
\color{black}{\em acknowledge}\color{black}	& *** & *** & *** & *** & *** & *** & ** & * & n.s. & n.s. & - & - & - & - & - & - & - & - & - & -\\
\color{black}{\em admit}\color{black}			& *** & *** & *** & *** & *** & *** & ** & * & n.s. & n.s. & n.s. & - & - & - & - & - & - & - & - & -\\
\color{black}{\em establish}\color{black}		& *** & *** & *** & *** & *** & *** & ** & * & n.s. & n.s. & n.s. &  ns & - & - & - & - & - & - & - & -\\
\color{blue}{\em be annoyed}\color{black}	& *** & *** & *** & *** & *** & *** & *** & ** & n.s. & n.s. & n.s. & n.s. & n.s. & - & - & - & - & - & - & -\\
\color{blue}{\em know}\color{black}		& *** & *** & *** & *** & *** & *** & *** & *** & n.s. & n.s. & n.s. & n.s. & n.s. & n.s. & - & - & - & - & - & -\\
\color{black}{\em confirm}\color{black}		& *** & *** & *** & *** & *** & *** & *** & *** & * & n.s. & n.s. & n.s. & n.s. & n.s. & n.s. & - & - & - & - & -\\
\color{blue}{\em discover}\color{black}			& *** & *** & *** & *** & *** & *** & *** & *** & * & n.s. & n.s. & n.s. & n.s. & n.s. & n.s. & n.s. & - & - & - & -\\
\color{blue}{\em see}\color{black}			& *** & *** & *** & *** & *** & *** & *** & *** & ** & n.s. & n.s. & n.s. & n.s. & n.s. & n.s. & n.s. & n.s. & - & - & - \\
\color{airforceblue}{\em be right}\color{black}			& *** & *** & *** & *** & *** & *** & *** & *** & ** & * & . & . & . & n.s. & n.s. & n.s. & n.s. & n.s. & - & - \\
\color{black}{\em prove}\color{black}		& *** & *** & *** & *** & *** & *** & *** & *** & **  & *  & * & * & * & n.s. & n.s. & n.s. & n.s. & n.s. & n.s. & - \\
\color{black}controls (\ref{control-good2})\color{black}		& *** & *** & *** & *** & *** & *** & *** & *** & ***  & ** & ** & ** & ** & . & n.s. & n.s. & n.s. & n.s. & n.s.  & n.s. \\

\bottomrule
\end{tabular}
\caption{P-values associated with pairwise comparison of inference ratings of the 20 clause-embedding predicates and the entailing controls in (\ref{control-good2}) using Tukey's method. `***' indicates significance at .0001, `**' at .01, `*' at .05, `.' marginal significance at .1, and `n.s.' indicates no significant difference in means. `Factive' predicates are given in darker blue, `veridical non-factive' ones in lighter blue, `non-factive' ones in brown and `apparently factive' ones in black.}\label{t-pairwise2}
\end{table}

%\begin{table}[h!]
%\setlength\tabcolsep{3pt}
%
%\centering
%\small
%
%\begin{tabular}{l l l l l l l l l l l l l l l l l l l l }
%\toprule
%
% &   \rot{\color{black}controls\color{black}}  & \rot{\color{brown}{\em pretend}\color{black}} & \rot{\color{brown}{\em think}\color{black}} & \rot{\color{brown}{\em suggest}\color{black}} &  \rot{\color{blue}{\em hear}\color{black}} &  \rot{\color{brown}{\em say}\color{black}} & \rot{{\em announce}} & \rot{{\em inform}} & \rot{\color{airforceblue}{\em demonstrate}\color{black}} & \rot{\color{black}{\em confess}\color{black}} & \rot{\color{blue}{\em reveal}\color{black}} & \rot{{\em acknowledge}} & \rot{{\em admit}} & \rot{\color{black}{\em establish}\color{black}}  & \rot{\color{blue}{\em be annoyed}\color{black}}  & \rot{\color{blue}{\em know}\color{black}}  & \rot{\color{black}{\em confirm}\color{black}} & \rot{\color{blue}{\em discover}\color{black}}  & \rot{\color{blue}{\em see}\color{black}}  & \rot{\color{airforceblue}{\em be right}\color{black}} \\
%\midrule
%
%\color{brown}{\em think}\color{black}		& *** & - & - & - & - & - & - & - & - & - & - & - & - & - & - & - & - & - & - \\
%\color{brown}{\em suggest}\color{black}			& *** & n.s. & - & - & - & - & - & - & - & - & - & - & - & - & - & - & - & - & - \\
%\color{blue}{\em hear}	\color{black}		& *** & *** & *** & - & - & - & - & - & - & - & - & - & - & - & - & - & - & - & - \\
%\color{brown}{\em say}\color{black}		& *** & *** & *** & *** & - & - & - & - & - & - & - & - & - & - & - & - & - & - & - \\
%\color{black}{\em announce}\color{black}		& *** & *** & *** & *** & *** & - & - & - & - & - & - & - & - & - & - & - & - & - & - \\
%\color{black}{\em inform}\color{black}		& *** & *** & *** & *** & *** & n.s. & - & - & - & - & - & - & - & - & - & - & - & - & - \\
%\color{airforceblue}{\em demonstrate}\color{black}		& *** & *** & *** & *** & *** & n.s. & n.s. & - & - & - & - & - & - & - & - & - & - & - & - \\
%\color{black}{\em confess}\color{black}		& *** & *** & *** & *** & *** & *** & * & n.s. & - & - & - & - & - & - & - & - & - & - & - \\
%\color{blue}{\em reveal}\color{black}		& *** & *** & *** & *** & *** & *** & ** & * & n.s. & - & - & - & - & - & - & - & - & - & - \\
%\color{black}{\em acknowledge}\color{black}	& *** & *** & *** & *** & *** & *** & ** & * & n.s. & n.s. & - & - & - & - & - & - & - & - & - \\
%\color{black}{\em admit}\color{black}			& *** & *** & *** & *** & *** & *** & ** & * & n.s. & n.s. & n.s. & - & - & - & - & - & - & - & - \\
%\color{black}{\em establish}\color{black}		& *** & *** & *** & *** & *** & *** & ** & * & n.s. & n.s. & n.s. &  ns & - & - & - & - & - & - & - \\
%\color{blue}{\em be annoyed}\color{black}	& *** & *** & *** & *** & *** & *** & *** & ** & n.s. & n.s. & n.s. & n.s. & n.s. & - & - & - & - & - & - \\
%\color{blue}{\em know}\color{black}		& *** & *** & *** & *** & *** & *** & *** & *** & n.s. & n.s. & n.s. & n.s. & n.s. & n.s. & - & - & - & - & - \\
%\color{black}{\em confirm}\color{black}		& *** & *** & *** & *** & *** & *** & *** & *** & . & n.s. & n.s. & n.s. & n.s. & n.s. & n.s. & - & - & - & - \\
%\color{blue}{\em discover}\color{black}			& *** & *** & *** & *** & *** & *** & *** & *** & * & n.s. & n.s. & n.s. & n.s. & n.s. & n.s. & n.s. & - & - & - \\
%\color{blue}{\em see}\color{black}			& *** & *** & *** & *** & *** & *** & *** & *** & * & n.s. & n.s. & n.s. & n.s. & n.s. & n.s. & n.s. & n.s. & - & - \\
%\color{airforceblue}{\em be right}\color{black}			& *** & *** & *** & *** & *** & *** & *** & *** & * & . & n.s. & n.s. & n.s. & n.s. & n.s. & n.s. & n.s. & n.s. & -  \\
%\color{black}{\em prove}\color{black}		& *** & *** & *** & *** & *** & *** & *** & *** & *  & *  & . & . & . & n.s. & n.s. & n.s. & n.s. & n.s. & n.s.  \\
%
%\bottomrule
%\end{tabular}
%\caption{P-values associated with pairwise comparison of inference ratings of the 20 clause-embedding predicates and the controls in (\ref{control-good2}) using Tukey's method. `***' indicates significance at .0001, `**' at .01, `*' at .05, `.' marginal significance at .1, and `n.s.' indicates no significant difference in means. `Factive' predicates are given in darker blue, `veridical non-factive' ones in lighter blue, `non-factive' ones in brown and `apparently factive' ones in black.}\label{t-pairwise2}
%\end{table}

However, not all `factive' and `veridical non-factive' predicates received inference ratings that were at ceiling, contrary to what is expected given their standard classification. First, inference ratings for the `factive' predicate {\em reveal} were significantly lower than for the `projective non-factive' predicate {\em prove}, the `veridical non-factive' predicate {\em be right} and the entailing controls in (\ref{control-good2}). Second, inference ratings for the `veridical non-factive' predicate {\em demonstrate} were significantly lower than that for the 5 `factive' predicates, the `veridical non-factive' predicate {\em be right}, the entailing controls in (\ref{control-good2}), as well as the `projective non-factive' predicates {\em prove, confirm, establish, admit} and {\em acknowledge}. These findings suggest that the content of the complement of the `factive' predicate {\em reveal} and of the `veridical non-factive' predicate {\em demonstrate} is not an entailment.

\subsection{Experiment 2b: Contradictoriness diagnostic for entailment}\label{s32}

This experiment explored whether the content of the complement of the 20 clause-embedding predicates is an entailment based on the contradictoriness diagnostic for entailment in (\ref{diag}b). Participants rated the contradictoriness of utterances of English sentences of the form {\em $\phi$ but not $\psi$}, where $\phi$ is an unembedded matrix sentence with a clause-embedding predicate and $\psi$ is its clausal complement. We collected gradient rather than binary contradictoriness ratings because we anticipated that participants' assessment of the contradictoriness of a speaker's utterance may depend on contextual factors: for instance, we assume that the extent to which  an utterance of the sentence in (\ref{announce3}) with the `projective non-factive' predicate {\em announce} is contradictory may depend on how trustworthy Mary is. 

\begin{exe}
\ex\label{announce3} Mary announced that she is pregnant, but she's not.
\end{exe}
We assume that when the content of $\phi$ entails the content of $\psi$, the contradictoriness of an utterance of the form {\em $\phi$ but not $\psi$} is not mitigated by contextual factors, and participants' contradictoriness ratings will be at ceiling. If, however, the contradictoriness of an utterance of the form {\em $\phi$ but not $\psi$} is mitigated by contextual factors, then the content of $\phi$ does not entail the content of $\psi$, and participants' contradictoriness ratings will be somewhere below ceiling.

\subsubsection{Methods}

\paragraph{Participants} 300 participants with U.S.\ IP addresses and at least 99\% of previous HITs approved were recruited on Amazon's Mechanical Turk platform (ages: 18-72, median: 35; 137 female, 162 male, 1 other). They were paid 75 cents for participating in the experiment.

\paragraph{Materials} In the target stimuli, the 400 predicate/clause combinations were combined with a random proper name subject and a {\em but-}clause that denied the truth of the content of the clausal complement. As shown by the sample stimuli in ({\ref{stims}), the target stimuli were presented to participants as utterances by named speakers. The proper names that realized the speakers, the subjects of the 20 predicates and the subjects of the complement clauses were all unique. The gender of the proper name subject of the predicate was distinct from the gender of the proper name in the complement clause, to ensure that the pronoun in the elliptical {\em but-}clause unambiguously referred to the individual referred to in the complement clause of the predicate.

\begin{exe}
\ex\label{stims}
\begin{xlist}
\ex {\bf Christopher:} {\em ``Melissa knows that Danny ate the last cupcake, but he didn't.''}
\ex {\bf Susan:} {\em ``Jerry pretended that Emma studied on Saturday morning, but she didn't.}
\end{xlist}
\end{exe}

The experiment also included eight control stimuli that were used to assess whether participants were attending to the task, whether they interpreted the task correctly, as well as to identify inference ratings that were at ceiling and at floor. (These stimuli were also presented to the participants as utterances by named speakers.) For the four non-contradictory control stimuli in (\ref{control-good}), we expected contradictoriness ratings at floor because the content of the second clause does not contradict the content of the first clause: for instance, in (\ref{control-good}d), the content that Vanessa is good at math does not entail the negation of the content that the speaker is not good at math. In contrast, we expected contradictoriness ratings at ceiling for the four contradictory control stimuli in (\ref{control-bad}) because the first clause of each of these stimuli entails the negation of the second clause: in (\ref{control-bad}c), for instance, the content of {\em Madison laughed loudly} entails that Madison laughed, i.e., the negation of {\em she didn't laugh}.

\begin{exe}
\ex\label{control-good} Non-contradictory control stimuli
\begin{xlist}
\ex Zack believes that I'm married, but I'm actually single.
\ex Tara wants me to cook for her and I'm a terrific cook.
\ex Frederick is both smarter and taller than I am.
\ex Vanessa is really good at math, but I'm not.
\end{xlist}
\ex\label{control-bad} Contradictory control stimuli
\begin{xlist}
\ex Dana has never smoked in her life and she stopped smoking recently.
\ex Hendrick's car is completely red and his car is not red.
\ex Madison laughed loudly and she didn't laugh.
\ex Sebastian lives in the USA and has never been to the USA.
\end{xlist}
\end{exe}

Each participant saw a random set of 28 stimuli: each set contained one target stimulus for each of the 20 predicates (each with a unique complement clause) and the same 8 control stimuli. Trial order was randomized.


\paragraph{Procedure.} Participants were told that they would read utterances made by a speaker and were asked to assess whether the speaker's utterance is contradictory. On each trial, participants read the speaker's utterance and then gave their response on a slider marked `definitely no' at one end (coded as 0) and `definitely yes' at the other (coded as 1), as shown in Figure \ref{f-trial-exp2}.

\begin{figure}[h!]
\begin{center}
\fbox{\includegraphics[width=13cm]{figures/contradictory-trial}}
\end{center}
\caption{A sample trial in Experiment 2b}\label{f-trial-exp2}
\end{figure}

To familiarize participants with the task, they first responded to the two familiarization stimuli in (\ref{train}): participants who rated (\ref{train}a) in the lower half of the scale or (\ref{train}b) in the upper half of the scale were given an explanation for why their answer was wrong. Participants could only advance to the 28 stimuli if they gave a plausible rating to the two training stimuli, i.e., a rating in the upper half of the scale for (\ref{train}b) and a rating in the lower half of the scale for (\ref{train}b).

\begin{exe}
\ex\label{train}
\begin{xlist}
\ex Drew is aware that Patty lives in Canada, but she doesn't.

\ex Drew thinks that Patty lives in Canada, but she doesn't.
\end{xlist}
\end{exe}

After responding to the 28 stimuli, participants filled out a short, optional survey about their age, their gender, their native language(s) and, if English is their native language, whether they are a speaker of American English (as opposed to, e.g., Australian or Indian English). To encourage them to respond truthfully, participants were told that they would be paid no matter what answers they gave in the survey.

\paragraph{Data exclusion}

Prior to analysis, the data from 19 participants who did not self-identify as native speakers of American English were excluded. For the remaining 281 participants, we inspected their responses to the 8 control stimuli: as expected, the group mean rating for the non-contradictory control stimuli in (\ref{control-good}) was at floor (.08)\footnote{The mean contradictoriness rating of the non-contradictory control stimulus that was of the same form as the target stimuli, namely (\ref{control-good}a) {\em Zack believes that I'm married, but I'm actually single}, was higher, at .17, than the mean ratings of the remaining three non-contradictory control stimuli (means: .05 or .06) and identical to that of the target stimuli with {\em think}.}  and the group mean rating for the contradictory control stimuli in (\ref{control-bad}) was at ceiling (.94). This finding shows that, overall, participants attended to and understood the task. The mean ratings of 12 participants were more than 3 standard deviations above the group mean for the non-contradictory control stimuli or below the group mean for the contradictory control stimuli. Closer inspection revealed that these participants' responses to the control stimuli were systematically higher or lower, suggesting that they did not attend to the task or interpreted the task differently. The data from these 12 participants were also excluded, leaving data from 269 participants (ages 18-72; median: 36; 128 female, 140 male, 1 other) and resulting in new group means of .06 for the non-contradictory control stimuli in (\ref{control-bad}) and .95 for the contradictory control stimuli in (\ref{control-good}b). 

\subsubsection{Results}

Each predicate received 269 ratings and each predicate/clause combination received between 3 and 22 ratings (mean: 13.5). The box plot in Figure \ref{f-veridicality-predicate} shows the contradictoriness ratings by predicate, collapsing over the 20 complement clauses, as well as the contradictoriness ratings of the non-contradictory controls in (\ref{control-good}) and of the contradictory controls in (\ref{control-bad}). The `plain non-factive' predicates {\em pretend, think, suggest} and {\em say} received among the lowest ratings overall, as expected from the assumption that the content of the complement of these predicates is not entailed. The only predicate whose mean and median contradictoriness ratings are at ceiling is the `veridical non-factive' predicate {\em be right} (mean: .93; median: .98); recall the group mean of .95 for the contradictory control stimuli in (\ref{control-bad}). This finding suggests that participants took target stimuli with {\em be right} to be highly contradictory. For target stimuli with the `projective non-factive' predicate {\em prove} and the `factive' predicate {\em know}, the median contradictoriness ratings are also at ceiling (.95 and .96, respectively), suggesting half of the participants considered target stimuli with these predicates to be highly contradictory. However, the larger box sizes, longer whisker lengths and lower mean contradictoriness ratings for {\em prove} and {\em know} suggest that participants did not consistently assess the target stimuli with these predicates to be contradictory. The same is true for the remaining `factive' and `veridical non-factive' predicates: the mean contradictoriness ratings are .83 for the `factive' predicate {\em know}, .81 for the `factive' predicate {\em see}, .78 for the `factive' predicate {\em discover}, .72 for the `veridical non-factive' predicate {\em demonstrate}, .64 for the `factive' predicate {\em be annoyed} and .63 for the `factive' predicate {\em reveal}.  

%          verb  Mean  YMin  YMax
%1  acknowledge 0.668 0.627 0.706
%2        admit 0.710 0.670 0.749
%3     announce 0.451 0.404 0.499
%4   be_annoyed 0.636 0.592 0.677
%5     be_right 0.935 0.916 0.951
%6      confess 0.668 0.626 0.708
%7      confirm 0.764 0.726 0.800
%8  demonstrate 0.719 0.681 0.756
%9     discover 0.781 0.745 0.816
%10   establish 0.719 0.679 0.755
%11        hear 0.239 0.205 0.275
%12      inform 0.470 0.423 0.515
%13        know 0.834 0.804 0.865
%14     pretend 0.219 0.183 0.258
%15       prove 0.854 0.828 0.881
%16      reveal 0.629 0.587 0.672
%17         say 0.367 0.323 0.413
%18         see 0.815 0.783 0.845
%19     suggest 0.258 0.221 0.294
%20       think 0.173 0.141 0.206

%  verb        Median
%   <chr>        <dbl>
% 1 acknowledge 0.780 
% 2 admit       0.810 
% 3 announce    0.360 
% 4 be_annoyed  0.720 
% 5 be_right    0.980 
% 6 confess     0.800 
% 7 confirm     0.890 
% 8 demonstrate 0.840 
% 9 discover    0.900 
%10 establish   0.830 
%11 hear        0.0900
%12 inform      0.470 
%13 know        0.960 
%14 pretend     0.0600
%15 prove       0.950 
%16 reveal      0.760 
%17 say         0.200 
%18 see         0.920 
%19 suggest     0.110 
%20 think       0.0300

\begin{figure}[h!]
\centering

\includegraphics[width=.75\paperwidth]{../results/2-veridicality2/graphs/boxplot-veridicality}

\caption{Box plot of contradictoriness ratings by predicate, collapsing across complement clauses, including the non-contradictory controls in (\ref{control-good}) (abbreviated `non-contrad.\ C') and the contradictory controls in (\ref{control-bad}) (abbreviated `contradictory C'). Grey dots indicate means and notches indicate medians. `Factive' predicates are given in darker blue, `veridical non-factive' ones in lighter blue, `plain non-factive' ones in brown and `projective non-factive' ones in black.}
\label{f-veridicality-predicate}
\end{figure}

To determine which clause-embedding predicates are such that the content of the complement is entailed, i.e., for which clause-embedding predicates the contradictoriness rating is  at ceiling, we conducted post hoc pairwise comparisons of the predicates and the contradictory control stimuli in (\ref{control-bad}) using Tukey's method (allowing for by-participant variability), using the \verb|lsmeans| package \citep{tukey} in R \citep{r}  (6456 data points total). The $p$-values for each comparison pair are shown in Table  \ref{t-pairwise}. These results suggest no difference in the contradictoriness ratings for the contradictory controls in (\ref{control-bad}) and the content of the complement of the `veridical non-factive' predicate {\em be right}. This finding is compatible with the assumption that the content of the complement of {\em be right} is entailed.


\begin{table}[h!]
\setlength\tabcolsep{3pt}

\centering
\small

\begin{tabular}{l l l l l l l l l l l l l l l l l l l l l}
\toprule

 &   \rot{\color{brown}{\em think}\color{black}} & \rot{\color{brown}{\em pretend}\color{black}} & \rot{\color{black}{\em hear}\color{black}} &  \rot{\color{brown}{\em suggest}\color{black}} &  \rot{\color{brown}{\em say}\color{black}} & \rot{{\em announce}} & \rot{{\em inform}} & \rot{\color{blue}{\em reveal}\color{black}} & \rot{\color{blue}{\em be annoyed}\color{black}} & \rot{{\em acknowledge}} & \rot{{\em confess}} & \rot{{\em admit}} & \rot{\color{airforceblue}{\em demonstrate}\color{black}}  & \rot{\color{black}{\em establish}}  & \rot{{\em confirm}}  & \rot{\color{blue}{\em discover}\color{black}} & \rot{\color{blue}{\em see}\color{black}}  & \rot{\color{blue}{\em know}\color{black}}  & \rot{{\em prove}} & \rot{\color{airforceblue}{\em be right}\color{black}} \\
\midrule
%\color{brown}{\em think}\color{black}		& x & - & - & - & - & - & - & - & - & - & - & - & - & - & - & - & - & - & - \\
\color{brown}{\em pretend}\color{black}		& n.s. & - & - & - & - & - & - & - & - & - & - & - & - & - & - & - & - & - & - & - \\
\color{black}{\em hear}\color{black}			& n.s. & n.s. & - & - & - & - & - & - & - & - & - & - & - & - & - & - & - & - & - & -\\
\color{brown}{\em suggest}	\color{black}		& * & n.s. & n.s. & - & - & - & - & - & - & - & - & - & - & - & - & - & - & - & - & -\\
\color{brown}{\em say}\color{black}		& *** & *** & *** & ** & - & - & - & - & - & - & - & - & - & - & - & - & - & - & - & -\\
\color{black}{\em announce}\color{black}		& *** & *** & *** & *** & * & - & - & - & - & - & - & - & - & - & - & - & - & - & - & -\\
\color{black}{\em inform}\color{black}		& *** & *** & *** & *** & ** & n.s. & - & - & - & - & - & - & - & - & - & - & - & - & - & -\\
\color{blue}{\em reveal}\color{black}		& *** & *** & *** & *** & *** & *** & *** & - & - & - & - & - & - & - & - & - & - & - & - & -\\
\color{blue}{\em be annoyed}\color{black}		& *** & *** & *** & *** & *** & *** & *** & n.s. & - & - & - & - & - & - & - & - & - & - & - & -\\
\color{black}{\em acknowledge}\color{black}		& *** & *** & *** & *** & *** & *** & *** & n.s. & n.s. & - & - & - & - & - & - & - & - & - & - & -\\
\color{black}{\em confess}\color{black}	& *** & *** & *** & *** & *** & *** & *** & n.s. & n.s. & n.s. & - & - & - & - & - & - & - & - & - & -\\
\color{black}{\em admit}\color{black}			& *** & *** & *** & *** & *** & *** & *** & * & . & n.s. & n.s. & - & - & - & - & - & - & - & - & -\\
\color{airforceblue}{\em demonstrate}\color{black}		& *** & *** & *** & *** & *** & *** & *** & ** & * & n.s. & n.s. &  n.s. & - & - & - & - & - & - & - & -\\
\color{black}{\em establish}\color{black}	& *** & *** & *** & *** & *** & *** & *** & ** & * & n.s. & n.s. & n.s. & n.s. & - & - & - & - & - & - & -\\
\color{black}{\em confirm}\color{black}		& *** & *** & *** & *** & *** & *** & *** & *** & *** & ** & ** & n.s. & n.s. & n.s. & - & - & - & - & - & -\\
\color{blue}{\em discover}\color{black}		& *** & *** & *** & *** & *** & *** & *** & *** & *** & *** & *** & n.s. & n.s. & n.s. & n.s. & - & - & - & - & -\\
\color{blue}{\em see}\color{black}			& *** & *** & *** & *** & *** & *** & *** & *** & *** & *** & *** & ** & ** & ** & n.s. & n.s. & - & - & - & -\\
\color{blue}{\em know}\color{black}			& *** & *** & *** & *** & *** & *** & *** & *** & *** & *** & *** & *** & *** & *** & n.s. & n.s. & n.s. & - & - & -\\
\color{black}{\em prove}\color{black}			& *** & *** & *** & *** & *** & *** & *** & *** & *** & *** & *** & *** & *** & *** & ** & . & n.s. & n.s. & -  & -\\
\color{airforceblue}{\em be right}\color{black}		& *** & *** & *** & *** & *** & *** & *** & *** & ***  & ***  & *** & *** & *** & *** & *** & *** & *** & ** & *  & -\\
\color{black}controls (\ref{control-bad})\color{black}		& *** & *** & *** & *** & *** & *** & *** & *** & ***  & ***  & *** & *** & *** & *** & *** & *** & *** & *** & ***  & n.s. \\

\bottomrule
\end{tabular}
\caption{P-values associated with pairwise comparison of contradictoriness ratings of clause-embedding predicates and the controls in (\ref{control-bad}) using Tukey's method. `***' indicates significance at .0001, `**' at .01, `*' at .05, `.' marginal significance at .1, and `ns' indicates no significant difference in means. `Factive' predicates are given in darker blue, `veridical non-factive' ones in lighter blue, `plain non-factive' ones in brown and `projective non-factive' ones in black.}\label{t-pairwise}
\end{table}

%\begin{table}[h!]
%\setlength\tabcolsep{3pt}
%
%\centering
%\small
%
%\begin{tabular}{l l l l l l l l l l l l l l l l l l l l }
%\toprule
%
% &   \rot{\color{brown}{\em think}\color{black}} & \rot{\color{brown}{\em pretend}\color{black}} & \rot{\color{blue}{\em hear}\color{black}} &  \rot{\color{brown}{\em suggest}\color{black}} &  \rot{\color{brown}{\em say}\color{black}} & \rot{{\em announce}} & \rot{{\em inform}} & \rot{\color{blue}{\em reveal}\color{black}} & \rot{\color{blue}{\em be annoyed}\color{black}} & \rot{{\em confess}} & \rot{{\em acknowledge}} & \rot{{\em admit}} & \rot{\color{black}{\em establish}\color{black}}  & \rot{\color{airforceblue}{\em demonstrate}}  & \rot{{\em confirm}}  & \rot{\color{blue}{\em discover}\color{black}} & \rot{\color{blue}{\em see}\color{black}}  & \rot{\color{blue}{\em know}\color{black}}  & \rot{{\em prove}}  \\
%\midrule
%%\color{brown}{\em think}\color{black}		& x & - & - & - & - & - & - & - & - & - & - & - & - & - & - & - & - & - & - \\
%\color{brown}{\em pretend}\color{black}		& n.s. & - & - & - & - & - & - & - & - & - & - & - & - & - & - & - & - & - & - \\
%\color{blue}{\em hear}\color{black}			& n.s. & n.s. & - & - & - & - & - & - & - & - & - & - & - & - & - & - & - & - & - \\
%\color{brown}{\em suggest}	\color{black}		& * & n.s. & n.s. & - & - & - & - & - & - & - & - & - & - & - & - & - & - & - & - \\
%\color{brown}{\em say}\color{black}		& *** & *** & *** & ** & - & - & - & - & - & - & - & - & - & - & - & - & - & - & - \\
%\color{black}{\em announce}\color{black}		& *** & *** & *** & *** & * & - & - & - & - & - & - & - & - & - & - & - & - & - & - \\
%\color{black}{\em inform}\color{black}		& *** & *** & *** & *** & ** & n.s. & - & - & - & - & - & - & - & - & - & - & - & - & - \\
%\color{blue}{\em reveal}\color{black}		& *** & *** & *** & *** & *** & *** & *** & - & - & - & - & - & - & - & - & - & - & - & - \\
%\color{blue}{\em be annoyed}\color{black}		& *** & *** & *** & *** & *** & *** & *** & n.s. & - & - & - & - & - & - & - & - & - & - & - \\
%\color{black}{\em confess}\color{black}		& *** & *** & *** & *** & *** & *** & *** & n.s. & n.s. & - & - & - & - & - & - & - & - & - & - \\
%\color{black}{\em acknowledge}\color{black}	& *** & *** & *** & *** & *** & *** & *** & n.s. & n.s. & n.s. & - & - & - & - & - & - & - & - & - \\
%\color{black}{\em admit}\color{black}			& *** & *** & *** & *** & *** & *** & *** & * & . & n.s. & n.s. & - & - & - & - & - & - & - & - \\
%\color{black}{\em establish}\color{black}		& *** & *** & *** & *** & *** & *** & *** & ** & * & n.s. & n.s. &  ns & - & - & - & - & - & - & - \\
%\color{airforceblue}{\em demonstrate}\color{black}	& *** & *** & *** & *** & *** & *** & *** & ** & * & n.s. & n.s. & n.s. & n.s. & - & - & - & - & - & - \\
%\color{black}{\em confirm}\color{black}		& *** & *** & *** & *** & *** & *** & *** & *** & *** & ** & ** & n.s. & n.s. & n.s. & - & - & - & - & - \\
%\color{blue}{\em discover}\color{black}		& *** & *** & *** & *** & *** & *** & *** & *** & *** & ** & ** & n.s. & n.s. & n.s. & n.s. & - & - & - & - \\
%\color{blue}{\em see}\color{black}			& *** & *** & *** & *** & *** & *** & *** & *** & *** & *** & *** & ** & ** & ** & n.s. & n.s. & - & - & - \\
%\color{blue}{\em know}\color{black}			& *** & *** & *** & *** & *** & *** & *** & *** & *** & *** & *** & *** & *** & *** & n.s. & n.s. & n.s. & - & - \\
%\color{black}{\em prove}\color{black}			& *** & *** & *** & *** & *** & *** & *** & *** & *** & *** & *** & *** & *** & *** & ** & n.s. & n.s. & n.s. & -  \\
%\color{airforceblue}{\em be right}\color{black}		& *** & *** & *** & *** & *** & *** & *** & *** & ***  & ***  & *** & *** & *** & *** & *** & *** & *** & ** & *  \\
%
%\bottomrule
%\end{tabular}
%\caption{P-values associated with pairwise comparison of contradictoriness ratings of predicates using Tukey's method. `***' indicates significance at .0001, `**' at .01, `*' at .05, `.' marginal significance at .1, and `ns' indicates no significant difference in means. `Factive' predicates are given in darker blue, `veridical non-factive' ones in lighter blue, `plain non-factive' ones in brown and `projective non-factive' ones in black.}\label{t-pairwise}
%\end{table}

However, the majority of `factive' and `veridical non-factive' predicates received contradictoriness ratings that were not at ceiling, contrary to what is expected given their standard classification. First, the contradictoriness ratings for all 5 `factive' predicates were significantly lower than for the contradictory controls and the `veridical non-factive' predicate {\em be right}. Also, the contradictoriness ratings of the `factive' predicates {\em reveal} and {\em be annoyed} were lower than those of the `factive' predicates {\em discover, see} and {\em know}, of the `projecting non-factive' predicates {\em prove, confirm} and {\em establish} and of the `veridical non-factive' predicate {\em demonstrate}. Second, the contradictoriness ratings of the `veridical non-factive' predicate {\em demonstrate} were significantly lower than for the `factive' predicates {\em see} and {\em know}, for the `projecting non-factive' predicate {\em prove}, for the `veridical non-factive' predicate {\em be right} and for the contradictory controls. These findings support an analysis of the content of the complement of the `factive' predicates {\em know, see, discover, be annoyed} and {\em reveal}, and that of the `veridical non-factive' predicate {\em demonstrate} as not entailed.

\subsection{Discussion}\label{s33}

In Experiments 2 we applied two standard diagnostics for entailment to the contents of the complements of 20 clause-embedding predicates with the goal of identifying whether the contents are entailed. Given the definition of entailment, we assumed that if the content of the complement of a predicate is entailed, responses under either diagnostic are at ceiling, i.e., statistically indistinguishable from the relevant controls; otherwise, the content is not entailed.\footnote{One can, of course, imagine other linking functions for the two entailment experiments. For instance, one could assume a linking function according to which the content of the complement of a predicate is entailed if and only if the mean response to stimuli with the predicate is above .8. However, such a linking function is incompatible with the definition of entailment according to which the content of $\phi$ does not entail the content of $\psi$ if there is one world in which the content of $\phi$ is true and the content of $\psi$ is false. Furthermore, it is unclear how a non-arbitrary threshold could be chosen: why .8 rather than .85 or .75? And, finally, such a linking function would still not support the standard assumption that the content of the complement of `factive' and `veridical non-factive' predicates is entailed and that of `plain non-factive' and `projective non-factive' predicates is not: with a threshold of .8 on Exp.~2b, for instance, the content of the complement of the `factive' predicates {\em reveal, discover} and {\em be annoyed} and of the `veridical non-factive' predicate {\em demonstrate} would not be considered to be entailed, but that of the `projective non-factive' predicate {\em prove} would be.} Table \ref{t-summary} summarizes the findings of the two experiments: `$\checkmark$' indicates that the strength of the inference to the content of the complement (Exp.~2a) or the contradictoriness rating (Exp.~2b) was at ceiling; `$\times$' indicates that the strength of the inference or the contradictoriness rating was significantly lower than ceiling. As shown, the `veridical non-factive' predicate {\em be right} is the only predicate for which the responses were at ceiling in both experiments. For the `factive' predicates {\em see, discover, know} and {\em be annoyed}, and for the `projective non-factive' predicates {\em prove} and {\em confirm}, the responses in Exp.~2b were significantly lower than ceiling. For all other predicates, responses in both experiments were significantly lower than ceiling


\begin{table}[h!]
%\setlength\tabcolsep{3pt}

\centering
\small

\begin{tabular}{r c c c c c c c c c c c c c c c c c c c c}

Exp.\ & \rot{\color{airforceblue}{\em be right}\color{black}} &  \rot{\color{blue}{\em see}\color{black}} & \rot{\color{blue}{\em discover}\color{black}} & \rot{\color{blue}{\em know}\color{black}} & \rot{\color{blue}{\em be annoyed}\color{black}} & \rot{\color{black}{\em prove}\color{black}} & \rot{\color{black}{\em confirm}\color{black}} & \rot{\color{black}{\em establish}\color{black}} & \rot{\color{black}{\em admit}\color{black}} & \rot{\color{black}{\em acknowledge}\color{black}} & \rot{\color{airforceblue}{\em demonstrate}\color{black}} &  \rot{\color{blue}{\em reveal}\color{black}} & \rot{\color{black}{\em hear}\color{black}} & \rot{\color{black}{\em confess}\color{black}} & \rot{\color{black}{\em inform}\color{black}} & \rot{\color{black}{\em announce}\color{black}} & \rot{\color{brown}{\em say}\color{black}} & \rot{\color{brown}{\em suggest} \color{black}} & \rot{\color{brown}{\em pretend}\color{black}} & \rot{\color{brown}{\em think}\color{black}} \\ 

\toprule

2a & $\checkmark$ & $\checkmark$& $\checkmark$& $\checkmark$& $\checkmark$& $\checkmark$& $\checkmark$ & $\times$ & $\times$ &  $\times$ & $\times$ & $\times$ & $\times$ & $\times$ & $\times$ & $\times$ & $\times$ & $\times$ & $\times$ & $\times$ \\

2b & $\checkmark$ & $\times$ & $\times$ &  $\times$ & $\times$ & $\times$ & $\times$ & $\times$ & $\times$ & $\times$ & $\times$ & $\times$ & $\times$ & $\times$ & $\times$ & $\times$ & $\times$ & $\times$ & $\times$ & $\times$\\

\bottomrule
\end{tabular}
\caption{Findings of Exps.~2a and 2b for the 20 clause-embedding predicates. `Factive' predicates are given in darker blue, `veridical non-factive' ones in lighter blue, `plain non-factive' ones in brown and `projective non-factive' ones in black. `$\checkmark$' indicates that responses were at ceiling;  `$\times$' indicates that responses were significantly lower than ceiling.}\label{t-summary}
\end{table}

For 14 of the 20 clause-embedding predicates, the findings of the two experiments converge. First, the findings of both experiments support an analysis of the content of the complement of the `veridical non-factive' predicate {\em be right} as entailed because responses in both experiments to stimuli with this predicates were at ceiling. Second, the findings of both experiments support an analysis of 13 predicates as not entailed because responses in both experiments to stimuli with these predicates were lower than ceiling:  these 13 predicates include the `plain non-factive' predicates {\em say, suggest, pretend} and {\em think}, the `projective non-factive' predicates {\em establish, admit, acknowledge, hear, confess, inform} and {\em announce}, but also the purportedly `factive' predicate {\em reveal} and the purportedly `veridical non-factive' predicate {\em demonstrate}. Thus, for {\em reveal} and {\em demonstrate}, the findings do not support the prevailing assumption that the content of the complement of these predicates is entailed. 

For the remaining 6 clause-embedding predicates, the findings of the two experiments differ: for the `factive' predicates  {\em see, discover, know} and {\em be annoyed}  and the `projective non-factive' predicates {\em prove} and {\em confirm}, the inference to the content of the complement was at ceiling in Exp.~2a, supporting an analysis of the content of the complement as entailed, but the contradictoriness ratings were below ceiling in Exp.~2b, supporting an analysis as not entailed. Thus, for these 6 `factive' and `projective non-factive' predicates, the findings of the entailment experiments are weaker but nevertheless critical with respect to the question of which clause-embedding predicates are such that the content of their complement is entailed: the experiments do not support the standard assumption that the contents of some of these predicates are entailed and others are not entailed. Rather, these 6 predicates are indistinguishable under the two entailment diagnostics.

This finding also bears on the debate that was introduced in section \ref{s1}, about whether the content of the complement of the emotive predicate {\em be annoyed} and the cognitive predicate {\em know} is entailed. Some authors, like \citet{klein1975,giannakidou1998,schlenker2003} and \citet{egre2008}, took examples like (\ref{emotive2}a) and (\ref{fis}b), repeated below for convenience, to show that the content is not entailed. Other authors, like \citet{gazdar79a} and \citet{abrusan2011}, did not take such examples to counter-exemplify an analysis of the content of the complement as entailed. These authors instead suggested that such examples are cases of free indirect discourse or are understood with a different concept of knowledge. Under this proposal, (\ref{emotive2}a), for instance, ``reports an attitude of a subject towards facts as perceived by him'' (\citealt[514]{abrusan2011}). As a consequence, ``the implications of [(\ref{emotive2}a)] do not have to be shared by the speaker''  ({\em ibid.}), which accounts for the observation that the content of the complement does not follow from (\ref{emotive2}a). Similarly, \citet[515]{abrusan2011} suggested that speakers or writers of examples like (\ref{fis}b) ``are understood as reporting a belief or feeling of ``knowledge'' on somebody's part'', but not necessarily on part of the speaker. 

\begin{exe}

\exi{(\ref{emotive2}a)} Falsely believing that he had inflicted a fatal wound, Oedipus regretted killing the stranger on the road to Thebes \hfill (\citealt{klein1975})

\exi{(\ref{fis}b)} Everyone knew that stress caused ulcers, before two Australian doctors in the early 80s proved that ulcers are actually caused by bacterial infection. \hfill (\citealt[501]{hazlett2010})

\end{exe}
The findings of our experiments replicate the two positions of the debate. On the one hand, the findings of Exp.~2a show that the strength of the inference to the truth of the content of the complement of {\em be annoyed} and {\em know} is at ceiling. On the other hand, the findings of Exp.~2b show that speakers who utter sentences with such predicates need not be (taken to be) committed to the truth of the content of the complement. Advocates of the position that the content of the complement of {\em be annoyed} and {\em know} is entailed might argue that the findings of Exp.~2b do not counter-exemplify the analysis of the content as entailed because the speakers in Exp.~2b were presented as not being committed to the truth of the content of the complement (e.g., Christopher: {\em ``Melissa knows/is annoyed that Danny ate the last cupcake, but he didn't.''}). This, in turn, may have resulted in participants understanding these utterances as ``reporting a belief or feeling of knowledge'' on the attitude holder's part. We agree that this is a possible interpretation and, therefore, the findings of our entailment experiments do not resolve the question of whether the content of the complement of {\em be annoyed} and {\em know} is entailed. Importantly, however, this interpretation would apply not only to the `factive' predicates we investigated in Exp.~2b, but also to the `non-factive' predicates {\em prove} and {\em confirm}, which patterned like the `factive' predicates {\em see, discover, know} and {\em be annoyed} on both entailment experiments. Thus, this interpretation would not support the standard distinction between the purportedly `factive' predicates {\em see, discover, know} and {\em be annoyed}, on the one hand, and the purportedly `non-factive' ones  {\em prove} and {\em confirm}, on the other.

The two entailment diagnostics gave different results for the contents of the complements of the 20 clause-embedding predicates. Specifically, we  observe that there were more lower-than-ceiling responses under the contradictoriness diagnostics in Exp.~2b than under the inference diagnostic in Exp.~2a. As a consequence, the findings of Exp.~2a support an analysis of the content of the complement as  entailed for more predicates than the findings of Exp.~2b. We hypothesize that this difference is due to a difference in task demand. In order to give a not-at-ceiling response under the inference diagnostic, participants had to imagine a world in which the statement that was presented as true (e.g., {\em Mark informed Sam / established that Sophie got a tattoo}) is true and in which the content of the complement (e.g., {\em Sophie got a tattoo}) is false. Evidence that this task is challenging comes from the observation that even trained researchers can fail to imagine such a world (see section \ref{s1}). By contrast, to give a not-at-ceiling response under the contradictoriness diagnostic, participants merely had to assess the speaker's utterance as not entirely contradictory. 

Does the observation that the two entailment diagnostics give different results mean that one of them is more suitable to diagnosing entailment?  Entailment is a semantic notion, and the two diagnostics differ in whether they force participants to consider contextual information. Specifically, on the contradictoriness diagnostic, participants must consider utterances by a speaker and the consistency of the epistemic state of this speaker, whereas, on the inference diagnostic, they consider true statements, not utterances, and no other epistemic agent. One might therefore be tempted to assume that the inference diagnostic is more suitable to diagnosing the semantic notion of entailment. However, in view of the aforementioned task demand of the inference diagnostic, it may be that the inference diagnostic may result in false positives (i.e., affirms an entailment relationship where there isn't one) whereas the contradictoriness diagnostic may result in false negatives (i.e., denies an entailment relationship where there is one). Furthermore, as discussed in \citealt{demarneffe-etal2012}, even judgments about what follows from utterances may be influenced by pragmatic information. For instance, the presumed trustworthiness of the attitude holder may influence not just contradictoriness ratings but also inference ratings.

Resolving the question of how to best diagnose entailment with theoretically untrained speakers is not possible in the scope of this paper. We note, however, that resolving this question is of central importance for research that makes use of the notion of entailment in the analysis of the meanings of sentences with clause-embedding predicates, like projection analyses, as we discuss in the next section.

\section{Discussion}\label{s4}

The experiments reported on in the previous two sections were designed to investigate whether the contents of the complements of 20 English clause-embedding predicates are projective (Exp.~1) and entailed (Exps.~2), with the goal of identifying whether the long-standing and widely-assumed categorical division between `factive' and `non-factive' predicates is empirically supported. Specifically, our experiments investigated projectivity and entailment for 5 `factive' predicates, namely {\em be annoyed, know, see, discover} and {\em reveal}, as well as for 15 `non-factive' predicates, namely the `plain non-factive' predicates {\em pretend, suggest, say} and {\em think}, the `veridical non-factive' predicates {\em be right} and {\em demonstrate}, and the `projective non-factive' predicates {\em acknowledge, admit, announce, confess, confirm, establish, hear, inform} and {\em prove}. Given the prevailing classifications of these predicates, we expected the contents of their complements to pattern alike with respect to projectivity and entailment. However, our experiments identified significant within-class variation in both projectivity and entailment:

\begin{itemize}[topsep=-1pt,itemsep=1pt]

\item {\bf `Factive' predicates:} The content of the complement of these predicates is expected to be projective and entailed. There is, however, no empirical support for the assumption that the content of the complement of {\em reveal} is entailed, and only partial support for the remaining 4 predicates. Furthermore, there is by-predicate projection variability.

\item {\bf `Plain non-factive' predicates:} The content of the complement of these predicates is expected to be neither projective nor entailed. However, only the content of the complement of {\em pretend} is not projective; that of the other 3 predicates is at least weakly projective.

\item {\bf `Veridical non-factive' predicates:} The content of the complement of these predicates is expected to be entailed but not projective. Contrary to expectation, the content of the complement of both predicates investigated is at least weakly projective, and there is no empirical support for the assumption that the content of the complement of {\em demonstrate} is entailed. 

\item {\bf `Projective non-factive' predicates:} The content of the complement of these predicates is expected to be projective but not entailed. However, for {\em prove} and {\em confirm}, there is support for an analysis of the content of the complement as entailed. Furthermore, there is by-predicate projection variability.

\end{itemize}

In sum, our experiments did not find empirical support for the assumption that the content of the complement of predicates in each of these four classes patterns alike. With respect to entailment, there is only partial support for many of the `factive' and `veridical non-factive' predicates that the content of the complement is entailed; for {\em reveal} and {\em demonstrate} there is no support at all, and hence these predicates cannot be assumed to be `factive' and `veridical non-factive', respectively. Furthermore, the experiments identified significant with-class projection variability, which calls into question the naturalness of these four classes of predicates. 

Do the experiments support a different classification of the 20 predicates under investigation into `factive' and (various types of) `non-factive' ones? To answer this question, consider Figure \ref{f-summary-categorical}, which plots the mean certainty rating of the contents of the complements in Exp.~1 on the x-axis and, on the y-axis, whether the content of their complement is entailed under, in the left panel (a), the inference diagnostic of Exp.~2a and, in the right panel (b), the contradictoriness diagnostic of Exp.~2b. 

\begin{figure}[h]

\begin{subfigure}{.5\textwidth}
\centering
\includegraphics[width=.35\paperwidth]{../results/5-projectivity-no-fact/graphs/projection-by-inferenceEntailment}
\caption{Inference diagnostic for entailment}
\end{subfigure}%
\begin{subfigure}{.5\textwidth}
\centering
\includegraphics[width=.35\paperwidth]{../results/5-projectivity-no-fact/graphs/projection-by-contradictorinessEntailment}
\caption{Contradictoriness diagnostic for entailment}
\end{subfigure}

\caption{Properties of the contents of the complements of 20 English clause-embedding predicates: on the x-axis, the mean certainty rating (with 95\% confidence intervals) and, on the y-axis, whether the content of the clausal complement is entailed under (a) the inference diagnostic of Exp.~2a or (b) the contradictoriness diagnostic of Exp.~2b. `Factive' predicates are given in darker blue, `veridical non-factive' ones in lighter blue, `plain non-factive' ones in brown and `projective non-factive' ones in black.}\label{f-summary-categorical}

\end{figure}

Does a class of `factive' predicates emerge from the experiments? Given the findings of the inference diagnostic for entailment (left panel of Figure \ref{f-summary-categorical}), there are four predicates for which the content of the clausal complement is both entailed and relatively highly projective, namely {\em discover, be annoyed, see} and {\em know}. Whether a predicate like {\em confirm, prove} or {\em be right} should count as `factive' depends on how projective the content of the complement needs to be in order for the predicate to be considered `factive': the contents of the complements of {\em confirm, prove} or {\em be right} are significantly less projective than that of {\em discover, be annoyed, see} and {\em know}, but that of {\em discover} is also significantly less projective than that of {\em know} and {\em be annoyed}. It is unclear where a non-arbitrary line could be drawn between predicates, given that the contents of the complements of {\em confirm, prove} and {\em be right} are not non-projective, just less projective. Furthermore, no class of `factive' predicates appears to emerge given the findings of the  contradictoriness diagnostic for entailment (right panel). In short, our experiments do not provide clear empirical support for a class of `factive' predicates: whether there is such a class depends on how entailment is best diagnosed and how projective the content of the complement of a clause-embedding predicate needs to be to count as `factive'.

Given the finding that the content of the complement of {\em pretend} is not projective, the experiments can be taken to support a class of `plain non-factive' predicates for which the content of the complement is neither entailed nor projective. Which other predicates belong to this class is an open question. There is also empirical support for a class of `projective non-factive' predicates, i.e., predicates for which the content of the complement is projective but not entailed. In fact, this class appears to be quite large, given that the content of the complement of many more predicates than previously assumed is at least weakly projective. Of course, which predicates are `projective non-factive' ones again depends on the question of how entailment is best diagnosed. Furthermore, this class would be quite heterogeneous, given the rampant by-predicate projection variability, which again calls into question the naturalness of such a class of predicates. Finally, our experiments do not provide support for a class of `veridical non-factive' predicates, i.e., predicates for which the content of the complement is entailed but not projective. For instance, the content of the complement of {\em be right} is entailed but also weakly projective, and the content of the complement of {\em pretend} is not projective (as noted above) but also not entailed. 

In sum, the experiments provide straightforward support neither for the standard classification of the 20 clause-embedding predicates into `factive' and (various types of) `non-factive' ones nor for an alternative classification. The main issues with supporting such a categorical classification are that it is not clear i) how entailment is best diagnosed and ii) how gradient projectivity can motivate a binary classification. 

Resolving these issues is of critical importance for research that builds on the distinction between `factive' and `non-factive' predicates, like formal analyses of projection, as introduced in section \ref{s1}. Both lexicalist analyses (e.g., \citealt{heim83,vds92}) and entailment-based analyses (e.g., \citealt{abrusan2011,abrusan2016,romoli2015,best-question}) are challenged by our findings. Lexicalist analyses, on which the content of the complement of `factive' predicates is lexically specified as a presupposition, must resolve which clause-embedding predicates are `factive'; entailment-based analyses, on which the projectivity of the content of the complement is derived from its status as entailed content, must resolve for which clause-embedding predicates the content of the complement is entailed. Furthermore, entailment-based analyses also need to confront the observation that there may be contents that are entailed but only weakly projective. For instance, given the findings of the inference diagnostic for entailment (see panel (a) of Figure \ref{f-summary-categorical}), the content of the complement of {\em be right} and {\em confirm} is entailed but only weakly  ({\em be right}) or mildly ({\em confirm}) projective. On Abrus\'an's (2011, 2016) analysis, entailed content is predicted to not project if projection is incompatible with the asserted content or an implicature of the assertion, or if the content addresses the grammatically signaled background question (\citealt[511, 532]{abrusan2011}). On  \citetpos{best-question} analysis, entailed content is predicted to not project if the content is not entailed by the Question Under Discussion. Whether these analyses can systematically account for entailed content that is less projective is a question for future research (see also \citealt{tbd-variability} on this point).

The findings of Exp.~1 also show that the empirical purview of projection analyses is broader than previously assumed, given that the contents of the complements of almost all of the 20 clause-embedding predicates emerged as projective, albeit to varying degrees. To date, there are only very few formal analyses of the projectivity of the content of the complement of `non-factive' predicates.  One attempt is presented in \citealt{schlenker10}, who proposed that {\em announce} is a ``part-time presupposition trigger'' (p.139): ``in some contexts, [{\em announce}, JT\&JD] does not entail the truth of its complement; in other contexts, it entails and {\em presupposes} the truth of the complement'' ({\em ibid.}). The notion of entailment adopted by Schlenker appears to be a context-dependent one, not the standard notion assumed here. We refer to his notion as `Schlenker-entailment': in context $c$, the content of sentence $\phi$ Schlenker-entails the content of sentence $\psi$ if and only the truth of $\psi$ follows from the truth of $\phi$ in $c$.  Thus, on Schlenker's proposal, a context in which the content of the complement of {\em announce} projects is a context in which a true assertion of the unembedded sentence with {\em announce} Schlenker-entails the content. Abstracting over contexts, it follows that if the content of the complement of {\em announce} is less projective than that of some other clause-embedding predicate $P$, we expect the content of the complement of {\em announce} to follow less strongly from true assertions of unembedded sentences with {\em announce} than from such assertions with $P$. This expectation is not borne out: the content of the complement of {\em announce} is less projective than that of {\em hear} (Exp.~1), but on both the inference diagnostic (Exp.~2a) and the contradictoriness diagnostic (Exp.~2b), the content of the complement of {\em announce} received significantly higher responses than that of {\em hear}. We conclude that the findings of our experiments are not predicted by \citetpos{schlenker10} proposal for the projectivity of the content of the complement of `non-factive' predicates like {\em announce}.

Another proposal to derive projectivity for `non-factive' predicates is presented by \citet{spector-egre2015}, who proposed (p.1736) that predicates like {\em tell, predict} and {\em announce}  are ambiguous between a `factive' lexical entry on which the content of the complement is entailed and presupposed, and thereby projective, and a `non-factive' one on which it is not. One concern with this proposal is that it is not clear where to draw lines between predicates: which predicates only have a `factive' lexical entry, which ones have both a `factive' and a `non-factive' one, which ones have only have a `non-factive' one? Given our finding that {\em pretend} is the only one of the 20 clause-embedding predicates explored for which the content of the complement is not projective, a second concern with this proposal is that it may require the assumption of widespread ambiguity among clause-embedding predicates. Finally, it is not clear whether the proposal can account for the observed by-predicate projection variability. For instance, the assumption that {\em discover} and {\em be annoyed} only have a `factive' lexical entry does not suffice to predict that the content of the complement of {\em discover} is less projective than that of {\em be annoyed}; likewise, the assumption that {\em inform} and {\em announce} have both a `factive' and a `non-factive' lexical entry does not suffice to predict that the content of the complement of {\em announce} is less projective than that of {\em inform}.

A third proposal is sketched by \citet{anand-hacquard2014}: these authors suggested that speakers may be taken to be committed to the content of the complement of so-called assertive predicates like {\em acknowledge, admit} and {\em confirm} ``because of the kind of discourse moves that these predicates report'' (p.74). Specifically, utterances of sentences with these predicates can be used to report the success of an uptake in a reported common ground: for example, an utterance of a sentence like {\em Kim confirmed that Mary is the murderer} can report that the content $p$ of the  complement, that Mary is the murderer, is accepted ``into the common ground of the reported discourse'' ({\em ibid.}). The authors suggested that ``[t]his acceptance of $p$ can easily bleed into the actual common ground, under the assumption that no subsequent move removed $p$ from the common ground'' (p.74f). The authors briefly entertained an extension of this proposal to utterances of sentences with {\em inform} and {\em announce}, which convey that the content of the complement is part of the reported common ground: ``to the extent that we identify ourselves with that context'' (p.76), the speaker and the addressee are committed to the truth of the content of the complement. While this proposal would need to be fleshed out in more detail to be predictive, we are sympathetic to the idea that the discourse functions of utterances of sentences with clause-embedding predicates plays a role in deriving the projectivity of the content of the complement. The findings of Exp.~1 suggest, however, that more fine-grained distinctions between predicates and their discourse functions would need to be made because there are significant differences between predicates that \citet{anand-hacquard2014} suggested share discourse functions: the content of the complement of {\em confirm} is less projective than that of {\em admit} and {\em acknowledge}, and that of {\em announce} is less projective than that of {\em inform}. In sum, there is currently no projection analysis on the market that adequately accounts for the projectivity of the content of the complement of `non-factive' predicates. 

Where do we go from here? The two main findings of this paper are that i) projectivity and entailment, as diagnosed in this paper, do not straightforwardly support a categorical classification of predicates into `factive' and `non-factive' ones, and ii) the empirical purview of projection analyses is broader than previously assumed because the contents of the complements of many `non-factive' clause-embedding predicates are projective. These main findings motivate at least two fruitful ways forward, ways that are entirely compatible with one another. The first way forward is to identify whether other experimental implementations of the two standard entailment diagnostics or an application of a different entailment diagnostic altogether can provide better empirical support for the prevailing classification of predicates into ones for which the content of the complement is entailed and ones for which it is not. If such support can be provided, it is possible to continue to derive the projectivity of entailed content through a lexicalist or an entailment-based projection analysis (with the caveat, as discussed above and in \citealt{tbd-variability}, that current analyses do not predict the observed projection variability among `factive' predicates). 

The second way forward is to investigate how factors other than entailment are implicated in the projectivity of the content of the complement of clause-embedding predicates (and of other utterance content more generally). Factors that have already been identified in the literature as influencing the projectivity of utterance content are summarized in (\ref{factors}): 

\begin{exe}
\ex\label{factors} Factors influencing the projectivity of utterance content
\begin{xlist}

\ex Contextual information, including common ground (e.g., \citealt{gazdar79a,gazdar79b,tonhauser-etal-eval})

\ex Prior probability of events / lexical content (e.g., \citealt{tbd-variability})

\ex Lexical meaning and discourse function of the attitude predicate (e.g., \citealt{anand-hacquard2014,tbd-variability})

\ex Information structure, as marked for instance by prosody (e.g., \citealt{cummins-rohde2015,tonhauser-salt26,djaerv-bacovcin-salt27})

\ex At-issueness (e.g., \citealt{brst-salt10,best-question,abrusan2011,tbd-variability,mahler-nels})

\ex Perceived degree of reliability and trustworthiness of the attitude holder (e.g., \citealt{schlenker10,demarneffe-etal2012})

\end{xlist}
\end{exe}

As mentioned above, research on projectivity has been largely limited to entailed content. Many of the factors in (\ref{factors}), however, have already been shown to influence the projectivity of content regardless of whether it is entailed. \citet{tbd-variability}, for instance, investigated the projectivity of the content of the complement of many `factive' predicates, but also of {\em establish, confess} and {\em reveal} (having assumed that the content is entailed), and found that the factors prior probability, lexical meaning and at-issueness influence projectivity. Furthermore, as discussed above, \citet{anand-hacquard2014} assumed that the lexical meaning of a clause-embedding predicate influences the projectivity of the content of the complement of some `non-factive' predicates, and \citealt{schlenker10} and \citealt{demarneffe-etal2012} assumed that the trustworthiness of the attitude holder matters for `non-factive' predicates. These observations give rise to the following research questions: Which factors influence the projectivity of utterance content, including the contents of complements of clause-embedding predicates, and do they influence entailed and non-entailed content differently? Addressing these questions is an exciting avenue for future research and the answers to these questions hold the key to understanding whether distinct projection analyses are needed for entailed and non-entailed content.

\section{Conclusions}\label{s5}

This paper investigated the long-standing and widely-adopted distinction between `factive' and `non-factive' predicates on the basis of experiments designed to explore projectivity and entailment for the contents of the complements of 20 English clause-embedding predicates. These experiments revealed that i) projectivity and entailment do not straightforwardly support a categorical classification of predicates into `factive' and `non-factive' ones, and that ii) the empirical purview of projection analyses is broader than previously assumed. We also observed that current projection analyses do not adequately predict the projectivity of the contents of the complements of `non-factive' predicates, primarily because such analyses have limited their attention almost entirely to the contents of the complements of `factive' predicates. Our research thereby provides impetus for empirically-driven investigations of factors implicated in the projectivity of entailed as well as non-entailed content and the development of projection analyses for non-entailed content.


%Figure \ref{f-summary-gradient}: inference rating and contradictoriness ratings are also not good predictors
%
%\begin{figure}[h]
%\begin{subfigure}{.5\textwidth}
%\centering
%\includegraphics[width=.35\paperwidth]{../results/5-projectivity-no-fact/graphs/projection-by-inference}
%\caption{Inference diagnostic for entailment}
%\end{subfigure}%
%\begin{subfigure}{.5\textwidth}
%\centering
%\includegraphics[width=.35\paperwidth]{../results/5-projectivity-no-fact/graphs/projection-by-contradictoriness}
%\caption{Contradictoriness diagnostic for entailment}
%\end{subfigure}
%\caption{Properties of the contents of the complements of 20 English clause-embedding predicates: on the x-axis, the mean certainty rating (with 95\% confidence intervals) and, on the y-axis, the mean inference rating from Exp.~2a (panel a) and the mean contradictoriness rating from Exp.~2b (panel b). `Factive' predicates are given in darker blue, `veridical non-factive' ones in lighter blue, `plain non-factive' ones in brown and `projective non-factive' ones in black.}\label{f-summary-gradient}
%
%\end{figure}

\appendix

\setcounter{table}{0}
\renewcommand{\thetable}{A\arabic{table}}

\setcounter{figure}{0}
\renewcommand{\thefigure}{A\arabic{figure}}

%\section{Citations}
%
%That both properties are ascribed to presuppositions, including the content of the complement of `factive' predicates, can be seen from the following quotes:
%
%\begin{itemize}[topsep=0pt,itemsep=-3pt,leftmargin=12pt]
%
%\item \citealt[66f.]{beaver01}: \citet[119-123]{gazdar79a} ``describes the inferences associated with factive verbs, definite descriptions, aspectual verbs, and clefts as being indefeasible in simple affirmative sentences'', i.e., ``entailments''.
%
%\item \citealt[355]{ccmg90}: ``A sentence can both entail and presuppose another sentence [...]. Thus, [{\em Joan realizes that syntax deals with sentence structure}] both entails and presupposes [{\em Syntax deals with sentence struture}]."
%
%\item \citealt[345]{vds92}: ``Note that [global accommodation, i.e., projection, JT\&JD] is what we would expect given the intuitive notion of presupposition as information taken for granted and note also that this explains the intuition that presuppositions [...] are entailed by their matrix sentence.''
%
%\item \citealt[3]{abbott06}: ``we will need to be careful to distinguish entailments that are presupposed from what I will call ``ordinary, simple entailments'', which are not also presuppositions.''
%
%\item \citealt[139]{schlenker10}: ``we obtain the pattern of inference which is characteristic of presuppositions: an entailment of the positive sentence is preserved under negation and in questions''
%
%
%\item \citealt[77]{anand-hacquard2014}: ``we will adopt the pragmatic view of presupposition triggering, according to which presuppositions are lexical entailments that are backgrounded based on pragmatic principles"
%
%\item {\bf \citealt[fn.7]{spector-egre2015}: Assuming that any presupposition of a sentence is also an entailment of this sentence, it follows that a predicate that is factive with respect to its declarative complement is always also veridical with respect to its declarative complement.}
%
%\end{itemize}

\section{20 complement clauses}\label{a-clauses}

The following clauses realized the complements of the predicates in Experiments 1 and 2:

\begin{enumerate}[leftmargin=3ex,itemsep=-2pt]

\begin{multicols}{2}

\item Mary is pregnant.
\item Josie went on vacation to France.
\item Emma studied on Saturday morning.
\item Olivia sleeps until noon.
\item Sophia got a tattoo.
\item Mia drank 2 cocktails last night.
\item Isabella ate a steak on Sunday.
\item  Emily bought a car yesterday.
\item  Grace visited her sister.
\item Zoe calculated the tip.

\columnbreak

\item  Danny ate the last cupcake.
\item  Frank got a cat.
\item  Jackson ran 10 miles.
\item  Jayden rented a car.
\item  Tony had a drink last night.
\item  Josh learned to ride a bike yesterday.
\item  Owen shoveled snow last winter.
\item  Julian dances salsa.
\item  Jon walks to work.
\item  Charley speaks Spanish.

\end{multicols}

\end{enumerate}


\bibliographystyle{/Users/tonhauser.1/Library/Latex/cslipubs-natbib}
\bibliography{/Users/tonhauser.1/Documents/bibliography}

\end{document}

